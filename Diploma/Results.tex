\chapter*{Заключение}
\addcontentsline{toc}{chapter}{Заключение}	% Добавляем его в оглавление
В работе были представлены результаты по проектированию оптических трактов экспериментальных станций первой очереди ЦКП <<СКИФ>>. Данные результаты используются для подготовки концептуального дизайна ЦКП <<СКИФ>>. Разработанный подход, а именно --- создание программной среды для проектирования, позволил создать надёжную базу для дальнейших расчётов с использованием современных вычислительных возможностей. По ссылке  \url{https://github.com/TrebAndrew/thesis_andrei/tree/dev} можно найти программный код, написанный на языке $\texttt{Python}$, с использованием библиотеки кода SRW для расчёта экспериментальных станций ЦКП <<СКИФ>>, а также результаты расчётов, не вошедшие в настоящую работу.

\textbf{Результаты расчётов}: 
\begin{itemize}
	\item для станции 1-4, для техники быстрой XAFS-спектроскопии, приведён один из возможных способов уширения спектра. Аналитически объяснена форма спектра, результат подтверждён численным моделированием;
	\item для станции 1-4 посчитан спектр ондулятора для XAFS-спектроскопии. Проведены расчёты распространения пучка через оптическую систему;
	\item приведены спектры излучения ондуляторов для станций 1-1 и 1-2, получены сечения пучка излучения на выходе из каждого оптических элементов;
	\item для станций 1-1 и 1-2 посчитаны удельные тепловые нагрузки на оптические элементы;
\end{itemize}

Новизна работы заключается, во-первых, в рассмотрении технически принципиально нового способа уширения спектра ондуляторного излучения, приведено его теоретическое объяснение и дано математическое моделирование излучения электронного пучка из описываемой магнитной структуры.

Во-вторых, в активном и структурированном использовании программного окружения и современных языков программирования по разработке рентгенооптических трактов эксперементальных станций синхротронного источника.

Ввиду высокой актуальности реализации проекта ЦКП <<СКИФ>> в планы по разработке экспериментальных станций входит: 

\begin{itemize}
	\item более детальное численное моделирование оптических элементов станций (учёт неидеальности элементов, учёт частичной когерентности излучения);
	\item создание математической модели, которая учитывает поправки к гармоническому магнитному полю неидеального ондулятора (фазовые ошибки);
	\item расчёт тепловых нагрузок на оптические элементы с учётом пространственного распределения мощности на поверхности оптических элементов;
	\item расчёт влияния устойчивости оптической системы к механическим подвижкам.
\end{itemize}