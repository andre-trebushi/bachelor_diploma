\chapter{Заключение}

В рукописи были представлены первые результаты по проектированию станций первой очереди ЦКП <<СКИФ>>. Приведённые результаты активно используются проектным офисом ЦКП <<СКИФ>> и при необходимости уточняются. Разработанный подход, а именно --- использование программной среды для проектирования, позволил создать надёжную базу для дальнейших расчётов с использование современных вычислительных возможностей. Ниже по пунктам приведены основные результаты: 

\begin{itemize}
	\item Приведены спектры излучения ондуляторов для станций 1-1 и 1-2, получены сечения пучка излучения на выходе каждого из оптических элементов.
	\item Для каждой из станций посчитаны удельные тепловые нагрузки на каждый из оптических элементов.
	\item Для станции 1-4 посчитан спектр ондулятора для EXAF спектроскопии. Просчитана оптическая система: сечение пучка на выходе из каждого оптических элементов.
	\item Для станции 1-4, для техники quick-EXAS спектроскопии, приведён один возможных способов уширения спектра. Аналитически объяснена форма спектра, результат подтверждён численным моделированием.
\end{itemize}

Новизна работы заключается, во-первых в активном и структурированном использовании программного окружения и современных языков программирования в разработке ренгенооптических трактов на синхротронном источнике, не было найдено подобных работ в российском научном сообществе. 

Во-вторых, в работе рассматривается моделирование перспективных источников излучения --- сверхпроводящих ондуляторов, что является новой вехой в источниках синхротронного излучения в 20-30-ые годы XXI века. 

В-третьих, рассмотрен технически принципиально новый способ уширения спектра ондуляторного излучения, приведено его теоретическое объяснение и дано математическое моделирование излучения электронного пучка из указанной магнитной структуры.

В дальнейшие планы по разработке пользовательских станций входит: 

\begin{itemize}
	\item Более детальное численное моделирование оптических элементов станций.
	\item Расчёт тепловых нагрузок на оптические элементы с учётом пространственного распределения мощности на поверхности оптического элемента.
	\item Расчёт влияния устойчивости оптической системы к механическим подвижкам.
	\item Создание математической модели реалистично описывающей распределение магнитных полей в ондуляторе.
\end{itemize}