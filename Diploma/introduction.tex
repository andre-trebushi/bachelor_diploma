\chapter*{Введение}							% Заголовок
\addcontentsline{toc}{chapter}{Введение}	% Добавляем его в оглавление

\newcommand{\actuality}{}
\newcommand{\aim}{\textbf{Целью}}
\newcommand{\tasks}{задачи}
\newcommand{\defpositions}{\textbf{Основные положения, выносимые на~защиту:}}
\newcommand{\novelty}{\textbf{Научная новизна}}
\newcommand{\influence}{\textbf{Научная и практическая значимость}}
\newcommand{\reliability}{\textbf{Степень достоверности}}
\newcommand{\probation}{\textbf{Апробация работы.}}
\newcommand{\contribution}{\textbf{Личный вклад.}}
\newcommand{\publications}{\textbf{Публикации.}}

{\actuality}
Данная работа посвящена разработке рентгенооптических трактов синхротронного источника СКИФ --- <<Сибирский кольцевой источник фотонов>>. За последние три десятилетия мир увидел активное развитие специализированных источников синхротронного излучения и соответствующих методов исследования вещества с использованием синхротронного излучения в рентгеновском диапазоне. Главные параметры излучения, который достигаются на данных установках является высокий поток фотонов, направленность излучения в малый телесный угол, когерентность. Эти параметры крайне необходимы для проведения качественных экспериментов с революционными результатами в области химии, биологии, материаловедении, медицины и многих других отраслях науки и техники.

Высокая востребованность данной работы заключается в том, что в отечественной науке наблюдается стагнация в области развития специализированных источников рентгеновского излучения. Проектируемый в Новосибирске синхротронный источник является первым на территории России специализированным источником с проектными параметрами не уступающими мировым установкам, а по некоторым данным с запасом превосходящих их, такие, например, как: MAX-IV, NSLS-II, PETRA-III, Diamond и д.р.

\textbf{Цель} данной работы --- разработка проекта станций первой очереди, вставными устройствами на которых являются сверхпроводящие ондуляторы. Это станции: 1-1 --- <<Микрофокус>>, 1-2 --- <<Структурная диагностика>>, 1-4 --- <<XAFS-спектроскопия и магнитный дихроизм>>.  

В цели проектирования входит ряд \textbf{задач}:
\begin{itemize}

	\item Расчёт ондуляторного излучения с помощью численного моделирования, получение спектров и сечений пучка из указанных устройств, максимально обективно описывающих реальное излучение.
	\item Разработка оптических трактов: расчёт тепловых нагрузок, расчёт спектров и сечений пучка после прохождение оптических элементов. 
	\item Разработка программного кода для реализации выше приведённых задач и удобному воспроизведению результатов расчётов любым участником проекта.

\end{itemize} % Характеристика работы по структуре во введении и в автореферате не отличается (ГОСТ Р 7.0.11, пункты 5.3.1 и 9.2.1), потому её загружаем из одного и того же внешнего файла, предварительно задав форму выделения некоторым параметрам

%% регистрируем счётчики в системе totcounter
\regtotcounter{totalcount@figure}
\regtotcounter{totalcount@table}       % Если поставить в преамбуле то ошибка в числе таблиц
\regtotcounter{TotPages}               % Если поставить в преамбуле то ошибка в числе страниц

%% на случай ошибок оставляю исходный кусок на месте, закомментированным
%Полный объём диссертации составляет  \ref*{TotPages}~страницу с~\totalfigures{}~рисунками и~\totaltables{}~таблицами. Список литературы содержит \total{citenum}~наименований.
%
