\chapter{Теоретической обоснование используемых численных методов}
\section{Излучение релятивистской электрона в синусоидальном магнитном поле}
В этой части мы дадим вывод излучения релятивистского электрона в $r\omega$-пространстве, движущегося в синусоидальном магнитном поле. Единственно приближение, которым мы будем пользоваться, --- прааксиальное приближение. Вывод интересен тем, что даёт наглядное представление о спектре частицы, угловом распределении интенсивности в зависимости от резонансной частоты. В заключении главы, будет приведён вывод распределения электромагнитного поля(в ближней зоне???) через потенциалы Лиенара-Вихерта, будет получен результат, который даст представление о методах используемых в численных симуляциях, на примере кода SRW. В наших рассуждениях мы следовали (Салдин Гелони)
\subsection{Уравнение движения электрона в ондуляторе}
Выведем спектр излучения из ондулятора. Вывод начнём с уравнения движение релятивистского электрона в магнитном поле.

\begin{equation}
	\vec{F} = e[\vec{v} \times \vec{B}],
\end{equation} 
где $e$ --- заряд электрона, а $\vec{v}$ и $\vec{B}$ скорость частицы и магнитное поле соответственно. Уравнение можно переписать в виде:

\begin{equation}
	\cfrac{d\vec{p}}{dt} = \cfrac{e}{\gamma m_e}[\vec{v} \times \vec{B}],
\end{equation}
где $\gamma$ --- лоренц фактор, появившийся из релятивистского импульса. Направим ось $z$ вдоль направления релятивистского движения электрона и введём магнитное поле в ондуляторе $B_0\cos(k_w z)$, направленное вдоль оси $y$, где $k_w$ связана с периодом ондулятора следующим образом $k_w = 2\pi/\lambda_w$. 

\begin{equation}
	\label{eq:eq_of_motion}
	\begin{cases}
		\cfrac{d^2 x}{dt^2} = - \cfrac{e B_0}{\gamma m_e}\cfrac{dz}{dt} \cos(k_w z)\\
		\cfrac{d^2 z}{dt^2} = \cfrac{e B_0}{\gamma m_e}\cfrac{dx}{dt} \cos(k_w z)
	\end{cases} 
\end{equation}

один раз интегрируя первой уравнение из системы с заменой $dz = \beta cdt$, где $\beta = \|\vec{v}\| /c$, можно получить: 

\begin{equation}
 	\label{eq:dx/dt}
	\cfrac{dx}{dt} = - \cfrac{eB_0}{\gamma m_ek_w} \sin(k_w z)
\end{equation}
Введём коэффициент ондуляторности --- $K = \cfrac{eB_0 \lambda}{2\pi m_ek_w}$, который показывает угол отклонения электрона от оси $z$(?????). 

Подставляя получившийся результат~\ref{eq:dx/dt} во второе уравнение системы~\ref{eq:eq_of_motion} и интегрируя с пределами интегрирования от $0$ до некоторого $z_0$, получим систему:

\begin{equation}
	\begin{cases}
	\label{eq:eq_of_motion_velocity}
		\cfrac{dx}{dt} = - \cfrac{Kc}{\gamma} \sin(k_w z)\\
		\cfrac{dz}{dt} = \beta c - \cfrac{K^2 c}{2 \gamma^2 \beta}\sin^2(k_w z)
	\end{cases} 
\end{equation}

Проинтегрировав оба уравнения(в каких пределах?), получим (Wiedemann),

\begin{equation}
	\begin{cases}
	\label{eq:eq_of_motion_trej}
		x = \cfrac{Kc}{с\gamma k_w \beta} \cos(k_w\overline{\beta}ct)\\
		z = \overline{\beta}ct + \cfrac{K^2}{8 \beta^2 \gamma^2 k_w}\sin(2k_w\overline{\beta}ct), 
	\end{cases} 
\end{equation}
где был введено обозначение $\overline{\beta}$, которое определяется как $\overline{\beta}c = \beta c(1 - \cfrac{K^2}{4 \beta^2 \gamma^2})$
Из~\ref{eq:eq_of_motion_velocity} видно, что продольная скорость испытывает о осцилляции с удвоенной частотой...
\subsection{Решение волнового уравнения в прааксиальном приближении}
Вывод спектра излучения будем проводить в $r\omega$-пространстве. Начнём с уравнений Максвелла в вакууме:
\begin{equation}
	\begin{cases}
		\nabla \cdot \vec{E} = 4\pi \rho\\
		\nabla \cdot \vec{E} = 0\\
		[\nabla \times \vec{E}] = -\cfrac{1}{c} \cfrac{d\vec{B}}{dt}\\
		[\nabla \times \vec{B}] = \cfrac{4\pi}{c} \vec{j} + \cfrac{1}{c} \cfrac{\partial\vec{E}}{\partial t}
	\end{cases} 
\end{equation}
Из уравнений тривиально можно получить неоднородное волновое уравнение(какая калибровка?): 
\begin{equation}
	\label{eq:inhomo_wave_eq_xt}
	\pdv[2]{\vec{E}}{t} + c^2 \nabla^2 \vec{E} = 4\pi c^2 \nabla \rho + 4\pi \pdv{\vec{j}}{t}
\end{equation}
Это же уравнение перепишем в $r\omega$-пространстве, определив преобразование Фурье следующим образом:
\begin{equation}
	\label{eq:Fourier_wt}
	\begin{array}{lcl}
		\vec{\widetilde{E}}(r, \omega) = \displaystyle\int\limits_{-\infty}^{\infty} dt \vec{E}(r, t)\exp[-i\omega t]\\
		\\
		\vec{E}(r, \omega) = \cfrac{1}{2\pi}\displaystyle\int\limits_{-\infty}^{\infty} d\omega \vec{\widetilde{E}}(r, t)\exp[i\omega t]
	\end{array}
\end{equation}
Применив к уравнению~\ref{eq:inhomo_wave_eq_xt}, получим:
\begin{equation}
	\omega^2 \vec{\widetilde{E}} + c^2 \nabla^2 \vec{\widetilde{E}} = 4\pi c^2 \nabla  \widetilde{\rho} - 4i\pi\omega\vec{\widetilde{j}}
\end{equation}
Перепишем это уравнение в приближении медленно меняющейся амплитуды в сравнение с частотой осцилляций, что есть $\vec{\widetilde{E}} =  \vec{\overline{E}}\exp[i\omega z/c]$, в приближении $\cfrac{\partial |E|}{\partial z} \ll \cfrac{\omega}{c}|E|$. Где временная зависимость разложена до нулевого порядка малости, исходя из уравнения~\ref{eq:eq_of_motion_trej}. Получим:
\begin{equation}
	\label{eq:wave_slow_vary}
	c^2\bigg(\nabla^2 \vec{\widetilde{E}} - \cfrac{2i\omega}{c}\pdv{\vec{\widetilde{E}}}{z}\bigg)\exp[i\omega z/c] = 4\pi c^2 \nabla  \widetilde{\rho} - 4i\pi\omega\vec{\widetilde{j}}
\end{equation}

Для электрона движущегося в вакууме ток и плотность заряда выражается через дельта-функцию Дирака:

\begin{equation}
	\begin{array}{lcl}
		\rho(r,t) = -e\delta(\vec{r}- \vec{r'}(t)) = -\cfrac{e}{v_z(z)}\delta(\vec{r}_{\bot}- \vec{r'}_{\bot}(z))\delta(\cfrac{s(z)}{v} - t)\\
		\vec{j}(r,t) = \vec{v}\rho(r,t)	
	\end{array}
\end{equation} 
В $r\omega$-пространстве: 
\begin{equation}
	\begin{array}{lcl}
		\widetilde{\rho}(r,\omega) = -\cfrac{e}{v_z(z)}\delta(\vec{r}_{\bot}- \vec{r'}_{\bot}(z))\exp[\cfrac{iws(z)}{v}]\\
		\widetilde{\vec{j}}(r,\omega) = \vec{v}\widetilde{\rho}(r,\omega)	
	\end{array}
\end{equation} 
Подставим фурье-образы плотности тока и заряда в уравнение~\ref{eq:wave_slow_vary}, (где производная по градиентному члену? добавить это)
\begin{equation}
	\label{eq:wave_eq}
	\begin{array}{lcl}
		\nabla^2 \vec{\widetilde{E}} - \cfrac{2i\omega}{c}\cfrac{\partial\vec{\widetilde{E}}}{\partial z} = 
		\cfrac{4\pi e}{v_z(z)} \exp[iw\bigg(\cfrac{s(z)}{v} - \cfrac{z}{c}\bigg)]
		\bigg(  
			\cfrac{i\omega}{c^2}\vec{v}(z)
			-\nabla\bigg) \delta(\vec{r}_{\bot} - \vec{r'}_{\bot}(z)) 
		
	\end{array}
\end{equation} 
Получившиеся уравнение является точным. Теперь мы можешь применить параксиальное приближении. 
\begin{equation}
	\label{eq:wave_slow_vary_parax}
	\begin{array}{lcl}
		\nabla_{\bot}^2 \vec{\widetilde{E}}_{\bot} - \cfrac{2i\omega}{c}\cfrac{\partial\vec{\widetilde{E}}_{\bot}}{\partial z} = 
		\cfrac{4\pi e}{v_z(z)} \exp[iw\bigg(\cfrac{s(z)}{v} - \cfrac{z}{c}\bigg)]\bigg(  
			\cfrac{i\omega}{c^2}\vec{v}_{\bot}(z) 
			-\nabla_{\bot}\bigg) \delta(\vec{r}_{\bot} - \vec{r'}_{\bot}(z)) 
	\end{array}
\end{equation} 
Вторая производная по z, появляющаяся из оператора Лапласа полагается много меньшим по сравнению с первой производной по $z$ в уравнении~\ref{eq:wave_slow_vary_parax} исходя из предположения медленно меняющейся амплитуды.

Перед нами неоднородное дифференциальное уравнение в частных производных, которое решается с помощью функции Грина. Для дифференциального оператора $\partial_t - k\nabla_{2D}^2$ функция Грина есть: $\cfrac{1}{4\pi kt}\exp[-\rho^2/4kt]$. В частности для уравнения~\ref{eq:wave_slow_vary_parax}
\begin{equation}
	\label{eq:Green_func}
	G(z_0 - z'; \vec{r}_{\bot 0} - \vec{r'}_{\bot}) = 
	- \cfrac{1}{4\pi (z_0 - z')}\exp[i\omega \cfrac{|\vec{r}_{\bot 0} - \vec{r'}_{\bot}|^2}{2c(z_0 - z')}]
\end{equation} 
Получим решение для функции распределения поля:

\begin{equation}
	\begin{array}{lcl}
		\vec{\widetilde{E}}_{\bot}(z_0,  \vec{r}_{\bot 0}, \omega) = -\cfrac{e}{c}  \displaystyle\int\limits_{-\infty}^{\infty}\int\limits_{-\infty}^{\infty} dz'd\vec{r'}\cfrac{1}{z_0 - z'}
		\bigg(\cfrac{i\omega}{c^2}\vec{v}_{\bot}(z')
		-\nabla'_{\bot}\bigg) \delta(\vec{r'}_{\bot} - \vec{r'}_{\bot}(z'))\times\\
		\exp[iw\bigg( \cfrac{|\vec{r}_{\bot 0} - \vec{r'}_{\bot}|^2}{2c(z_0 - z')} +\cfrac{s(z')}{v} - \cfrac{z'}{c} \bigg)]
	\end{array}	
\end{equation}
Проинтегрировав по $d\vec{r'}$ получим общее решение уравнения~\ref{eq:wave_eq} :
\begin{equation}
	\label{eq:field_in_parax}
	\begin{array}{lcl}
		\vec{\widetilde{E}}_{\bot}(z_0,  \vec{r}_{\bot 0}, \omega) = -\cfrac{i\omega e}{c^2}  \displaystyle\int\limits_{-\infty}^{\infty} dz'
		\cfrac{1}{z_0 - z'}
		\bigg(\cfrac{\vec{v}_{\bot}(z')}{c}
		- \cfrac{\vec{r}_{\bot 0} - \vec{r'}_{\bot}(z')}{(z_0 - z')}\bigg)\times\\
		\exp[iw\bigg(\cfrac{|\vec{r}_{\bot 0} - \vec{r'}_{\bot}(z')|^2}{2c(z_0 - z')} + \cfrac{s(z')}{v} - \cfrac{z'}{c} \bigg)]
	\end{array}	
\end{equation}
Что есть распределение электромагнитного поля в точке наблюдения $\vec{r}_0$.

\subsection{Излучение планарного ондулятора}
В этой секции мы рассмотрим излучение планарного ондулятора использую наши предыдущие результаты~\ref{eq:field_in_parax} и~\ref{eq:eq_of_motion_trej}. Сперва проанализируем получившиеся распределение поля~\ref{eq:field_in_parax}: в случае ондулятора, член $(z_0 - z')^{-1}$ можно разложить около точки $z'$, так как размер ондулятора много меньше чем расстояние, с которого мы наблюдаем излучение: $\lambda_w N \ll z_0$, где $N$ число периодов ондулятора.

Воспользовавшись решениями~\ref{eq:eq_of_motion_velocity} и~\ref{eq:eq_of_motion_trej} и помня $\vec{r}_{\bot 0}/z_0 = \vec{\theta}$ преобразуем уравнение~\ref{eq:field_in_parax} к виду:
\begin{equation}
	\label{eq:field_in_parax}
	\begin{array}{lcl}
		\vec{\widetilde{E}}_{\bot}(z_0,  \vec{r}_{\bot 0}, \omega) =
		\cfrac{i\omega e}{c^2z_0} \exp[i\cfrac{\omega \theta^2 z_0}{2c}]
	 	\displaystyle\int\limits_{-\lambda_w N/2}^{\lambda_w N/2} dz'\exp[i\Phi_T]
		\bigg(\cfrac{K}{\gamma}\sin(k_w z)\vec{e_x} + \vec{\theta}\bigg)
	\end{array}	
\end{equation}
Здесь мы отбросили члены первого и большего порядка малости по $1/z_0$. Где за $\Phi_T$ мы обозначили:
\begin{equation}
	\Phi_T = 
	\bigg(\cfrac{\omega}{2c\widetilde{\gamma}^2} + 
	\cfrac{\omega\vec{\theta}^2}{2c}\bigg)z' - 
	\cfrac{K^2}{8\gamma^2}\cfrac{\omega}{k_w c}\sin(2k_wz') - \cfrac{K{\theta_x}}{\gamma}\cfrac{\omega}{k_w c}\cos(k_w z'),
\end{equation}
где $\widetilde{\gamma} = \cfrac{\gamma}{\sqrt{1 + K^2/2}}$.

Пределы интегрирования ограничили по длиной ондулятора от $-\lambda_w N/2$ до $\lambda_w N/2$, считая вклад в излучение от ондулятора доминирующим надо остальными вкладами от соответствующих участков траектории. На это шаге уже можно заметить, что излучение на оси будет линейно поляризованно, это есть вклад члена с током, вклад же плотности заряда или градиентный член, даёт вариацию поляризации, при наблюдении под некоторым углом $\theta$ к оси.

Если переписать~\ref{eq:field_in_parax} в следующе виде:
\begin{equation}
		\label{eq:field_in_parax_Bessel}
		\begin{array}{lcl}
			\vec{\widetilde{E}}_{\bot}(z_0,  \vec{r}_{\bot 0}, \omega) =
			\cfrac{i\omega e}{c^2z_0} \exp[i\cfrac{\omega \theta^2 z_0}{2c}]
			\displaystyle\sum_{m,n=-\infty}^{+\infty}
			J_m\bigg(-\cfrac{K^2}{8\gamma^2}\cfrac{\omega}{k_w c}\bigg)
			J_n\bigg(-\cfrac{K{\theta_x}}{\gamma}\cfrac{\omega}{k_w c}\bigg)\times\\
			\exp[\cfrac{i\pi n}{2}]
			\displaystyle\int\limits_{-\lambda_w N/2}^{\lambda_w N/2} dz'\exp[i(2m + n)k_wz']
			\bigg(\cfrac{K}{2i\gamma}\big(\exp[2ik_w z'] - 1\big)\vec{e_x} + \vec{\theta}\exp[ik_w z']\bigg)\times\\
			\exp[i\bigg(k_w \cfrac{\Delta\omega}{\omega_r} + 
			\cfrac{\omega\vec{\theta}^2}{2c}\bigg)z'],
		\end{array}	
\end{equation}
Где мы ввели $\omega = \omega_r + \Delta\omega$, $\omega_r = 2c\widetilde{\gamma}^2k_w$ и использовали формулу Якоби — Ангера:
\begin{equation}
	\begin{array}{lcl}
		\exp[iz\cos(\theta)] = 
		\displaystyle\sum\limits_{n =-\infty}^{\infty}
		i^n J_n(z)\exp[in\theta]\\	
		\exp[iz\sin(\theta)] = 
		\displaystyle\sum\limits_{n =-\infty}^{\infty}
		J_n(z)\exp[in\theta]
	\end{array}	
\end{equation}




До сих пор мы пользовались только двумя приближениями, --- медленно меняющейся амплитудой и параксиальным приближением, теперь можем воспользоваться следующим параметром --- количеством периодов ондулятора, --- $N$. Для этого обратим внимание на первой слагаемое в фазовом множителя под интегралом, и заметим, что если $k_w \cfrac{\Delta\omega}{\omega_r} + 
\cfrac{\omega\vec{\theta}^2}{2c} \ll k_w$, то фаза меняется медленно на одном периоде и эта фаза не занулит интеграл. Отметим, что для резонанса условия должны выполняться по отдельности, т.е. $\cfrac{\Delta\omega}{\omega_r} \ll 1$ и $\cfrac{\omega\vec{\theta}^2}{2c} \ll 1$, последнее даёт углы наблюдения вблизи резонанса: $\theta^2 \ll \cfrac{1}{\widetilde{\gamma}^2}$. Теперь необходимо обратить внимание на аргументы функций Бесселя, а именно: 
\begin{equation}
	\begin{array}{lcl}
		u = -\cfrac{K^2}{8\gamma^2}\cfrac{\omega}{k_w c}\\
		v = -\cfrac{K{\theta_x}}{\gamma}\cfrac{\omega}{k_w c} = - \cfrac{K{\theta_x}}{\gamma}
		\bigg(1 + \cfrac{\Delta\omega}{\omega_r}\bigg)2\widetilde{\gamma}^2 \lesssim
		\cfrac{2K{\theta_x}\widetilde{\gamma}}{\sqrt{1 + K^2/2}} \lesssim \theta_x\widetilde{\gamma} \ll 1
	\end{array}	
\end{equation}
Зная, что $J_\alpha(x) \thicksim \displaystyle\sum\limits_{n =0}^{\infty} x^{2\beta + \alpha} $, видим, что вклад нулевого порядка по $\theta_x\widetilde{\gamma}$, т.е. $J_\alpha(x) \thicksim 1$ даёт только функция Бесселя с индексом $n = 0$. Здесь мы пока не учитываем градиентный член пропорциональный $\vec{\theta}$, таким образом из оставшихся фазовых множителей можно выписать условия на индекс $m$. Они определяются нулями в аргументах соответствующих фаз или $m = -1$ и $m = 0$, оба оставшихся члена пропорциональны $\cfrac{K}{\gamma}$. 

Теперь вернёмся к градиентному члену, вклад от которого занулиться при усреднении по длине ондулятора при $n = 0$, этот вклад даст ненулевой вклад при $n = 1 - 2m$, т.о. в ход пойдут следующие члены разложения $J_m(v)$. Однако, помня интересующий нас диапазон углов, члены разложения будут порядка $\theta_x v^m$, очевидно, что их вклады пренебрежимо малы по сравнению с вкладами токового члена $\vec{e}_x$. Учитывая выше сказанные приближения, перепишем~\ref{eq:field_in_parax_Bessel}
\begin{equation}
	\label{eq:field_dist_in_integral}
	\begin{array}{lcl}
		\vec{\widetilde{E}}_{\bot}(z_0,  \vec{r}_{\bot 0}, \omega) =
		\cfrac{\omega e}{2c^2z_0}\cfrac{K}{\gamma}\exp[i\cfrac{\omega \theta^2 z_0}{2c}]
		\bigg(J_1(v) - J_0(v)\bigg)\vec{e}_x\times\\
		\\
		\displaystyle\int\limits_{-\lambda_w N/2}^{\lambda_w N/2} dz'
		\exp[i\bigg(k_w \cfrac{\Delta\omega}{\omega_r} + 
		\cfrac{\omega\vec{\theta}^2}{2c}\bigg)z'],
	\end{array}	
\end{equation}
Интеграл легко берётся:
\begin{equation}
	\label{eq:field_dist_nonNorm}
	\begin{array}{lcl}
		\vec{\widetilde{E}}_{\bot}(z_0,  \vec{r}_{\bot 0}, \omega) =
		\cfrac{\omega eL}{c^2z_0}\cfrac{K}{\gamma}\exp[i\cfrac{\omega \theta^2 z_0}{2c}]
		\sinc[\bigg(k_w \cfrac{\Delta\omega}{\omega_r} + 
		\cfrac{\omega\vec{\theta}^2}{2c}\bigg)L/2]\vec{e}_x ,
	\end{array}	
\end{equation}
где введено обозначение: $A_{JJ} = J_1(v) - J_0(v)$. 

В следующем параграфе мы займёмся выводном влияния конченого эмиттанса на распределение излучения, чтобы облегчить выкладки мы введём нормализованные единицы.
\begin{equation}
	\label{eq:norm_units}
	\begin{array}{lcl}
		\hat{E}_{\bot} = \cfrac{c^2z_0\gamma \widetilde{E}_{\bot}}{e\omega KLA_{JJ}}\\
		\hat{\theta} = \theta\sqrt{\cfrac{\omega L}{c}}\\
		\hat{z} = \cfrac{z}{L} ,
	\end{array}	
\end{equation}
а также, 
\begin{equation}
	\hat{C} = CL = 2\pi N\cfrac{\Delta\omega}{\omega_r}
\end{equation}
Теперь уравнения~\ref{eq:field_dist_nonNorm} и~\ref{eq:field_dist_in_integral} могут быть переписаны в нормализованных единицах.
\begin{equation}
	\label{eq:field_dist_in_integral}
	\begin{array}{lcl}
		\hat{E}_{\bot} = e^{i\Phi}
		\displaystyle\int\limits_{-1/2}^{1/2} dz'
		\exp[i\bigg(\hat{C} + 
		\cfrac{\hat{\theta}^2}{2}\bigg)z'],
	\end{array}	
\end{equation}

\begin{equation}
	\label{eq:field_dist_Norm}
	\begin{array}{lcl}
		\hat{E}_{\bot} = e^{i\Phi}
		\sinc\bigg(\cfrac{\hat{C}}{2} + 
		\cfrac{\hat{\theta}^2}{4}\bigg),
	\end{array}	
\end{equation}
\begin{figure}
	\centering  
	\begin{minipage}{0.49\textwidth}
		\centering
		\includegraphics[width=\textwidth]{pic/angleC_neg.pdf}
		\caption{}
		\label{fig:angle_dist_C_neg}
	\end{minipage}\hfill
	\begin{minipage}{0.49\textwidth}
		\centering
		\includegraphics[width=\textwidth]{pic/angleC_pos.pdf}
		\caption{}
		\label{fig:angle_dist_C_pos}
	\end{minipage}    
\end{figure}
На рис.~\ref{fig:angle_dist_C_neg} и рис.~\ref{fig:angle_dist_C_pos} изображены угловые распределения излучения. Из них можно понять, что если сдвижка по спектру идёт в область меньших частот, то условие резонанса удовлетворяется на других углах и проинтегрированные по даст некоторую интенсивность. Если же сдвигаться по спектру в область более высоких частот, что условие резонанса на углах не будет выполняться и интенсивность быстро упадёт. На рис.~\ref{fig:spec_integrate_angle} представлен проинтегрированный по углам $\hat{\theta}$ спектр.
\begin{figure}[htbp]
	\centering
	\includegraphics[width=.99\textwidth]{{pic/spec_integ_ang}.pdf}
	\caption{} 
	\label{fig:spec_integrate_angle}
\end{figure}

\subsection{Излучение клинообразного ондулятора}
В этой секции мы рассмотрим излучение из ондулятора специальной конструкции, который может дать широкий спектр. Идея в том, что разбить ондулятор на несколько секций с различным $K$ в каждой из них. Такая расстановка в первом приближении должна дать набор резонансов, которые должны сложиться в один сплошной спектр. Более детальное рассмотрение покажет, что в зависимости от корреляции фазы электрона между этими сегментами, могут проявляться интерференционные эффекты, которые в значительной степени будут изменять форму спектра. В нашем рассмотрении мы покажем влияние указанных вкладов для случая скачкообразного изменения поля, а также для классического случая клинообразного, т.н. зарубежной литературе tapered undulator.

Свои выкладки начнём с модифицированного интеграла~\ref{eq:field_dist_in_integral}, 
\begin{equation}
	\begin{array}{lcl}
	\vec{\widetilde{E}}_{\bot}(z_0,  \vec{r}_{\bot 0}, \omega) =
	\cfrac{\omega eA_{JJ}}{2c^2z_0}\cfrac{K}{\gamma}
	\displaystyle\int\limits_{-\lambda_w N/2}^{\lambda_w N/2} dz'
	\exp[iCz'] 	\vec{e}_x,
	\end{array}	
\end{equation} 
Здесь, для краткости выкладок, излучения мы сморим на оси, т.е. $\theta = 0$. В случае секционного ондулятора коэффициент ондуляторности меняется вдоль ондулятора, поэтому $K = K_0 + n\Delta K$, а также $C = C_0 + n\Delta C$, где $n$ --- это номер секции. $\Delta {C}$ введено следующим образом, помня $\omega_r = 2c\widetilde{\gamma}^2k_w$:
\begin{equation}
	C =k_w\cfrac{\Delta \omega}{\omega_r} = \cfrac{\Delta \omega_r}{2c\gamma}\bigg(1 + \cfrac{(K_0 + n\Delta K)^2}{2}\bigg) \approx \cfrac{\Delta \omega_r}{2c\gamma}\bigg(1 + \cfrac{K^2_0}{2}(1 + \cfrac{n\Delta K}{K_0})\bigg) = C_0 + \Delta C
\end{equation} 
Секций, для условности, мы возьмём пять, и для удобства нумерацию будем вести $-2, -1, ... , 2$. Поэтому интеграл перепишеться в виде:
\begin{equation}
	\begin{array}{lcl}
	\vec{\widetilde{E}}_{\bot}(z_0,  \vec{r}_{\bot 0}, \omega) =
	\cfrac{\omega eA_{JJ}}{2c^2 \gamma z_0}
	\displaystyle\sum\limits_{n =-2}^{2}(K_0 + n\Delta K)
	\displaystyle\int\limits_{(2n + 1)L_s/2}^{(2n - 1)L_s/2} dz'
	\exp[i(C_0 + n\Delta C)z']	\vec{e}_x,
	\end{array}	
\end{equation} 
Взяв интеграл, получим:
\begin{equation}
	\begin{array}{lcl}
	\vec{\widetilde{E}}_{\bot}(z_0,  \vec{r}_{\bot 0}, \omega) =
	\cfrac{\omega eA_{JJ}}{2c^2 \gamma z_0}
	\displaystyle\sum\limits_{n =-2}^{2}(K_0 + n\Delta K)
	\sinc(\hat{C}/2)e^{in({C}_0 + n\Delta {C})L}	\vec{e}_x,
	\end{array}	
\end{equation} 
Возведя в квадрат получим интенсивность:
\begin{equation}
	\begin{array}{lcl}
	{\widetilde{I}} =
	\bigg(\cfrac{\omega eA_{JJ}}{2c^2 \gamma z_0}\bigg)^2\bigg[
	\displaystyle\sum\limits_{n =-2}^{2}(K_0 + n\Delta K)^2\sinc^2(\hat{C_0} + n\Delta \hat{C}/2) + \\
		
	\displaystyle\mathop{\sum\limits_{n, m =-2}^{2}}_{n \neq m}K^2_0\bigg(1 + n\cfrac{\Delta K}{K_0} + m\cfrac{\Delta K}{K_0}\bigg)
	\sinc^2(\hat{C}/2)e^{i(n-m)\hat{C}_0 + (n^2 - m^2)\Delta \hat{C}}\bigg],
	\end{array}	
\end{equation} 
Данное выражение можно проинтерпретировать следующим образом: первая сумма есть сумма сдвинутых по соответствующим резонансам $\sinc^2$ функций, вторая сумма отображает интерференцию между различными секциями ондулятора, и как кажется автору нежелательна. Данная комбинация приводит к хаотичным колебаниями в спектре, как показано на рис.~\ref{fig:section_und_analitics} пунктирными линиями, черной линей отмечана сумма $\sinc^2$ функций без учёта интерференционных слагаемых.
\begin{figure}
	\centering  
	\begin{minipage}{0.49\textwidth}
		\centering
		\includegraphics[width=\textwidth]{pic/spec_from_sec_und.pdf}
		\caption{}
		\label{fig:section_und_analitics}
	\end{minipage}\hfill
	\begin{minipage}{0.49\textwidth}
		\centering
		\includegraphics[width=\textwidth]{pic/spec_SRW.pdf}
		\caption{}
		\label{fig:section_und_SRW}
	\end{minipage}    
\end{figure}
На рис.~\ref{fig:section_und_SRW} показан характерный спектр секционного ондулятора посчитанного при помощи симуляционного кода SRW. Сравнение формы синей пунктирной линии на рис.~\ref{fig:section_und_analitics} и кривой на рис.~\ref{fig:section_und_SRW} показывает, что были сделаны правильные предположения в нашей аналитической модели, происходит интерференция между различными частями ондулятора.
Один из возможных путей, чтобы избавиться от интерференционных слагаемых в спектре ондуляторного излучения, добавить произвольную фазу между секциями ондулятора. Дело в том, что данные вычисления проводились для ондного электрона, если если мы хотим получить спектр, который получиться в точке наблюдения, то можно понять, что спектры на рис.~\ref{fig:section_und_analitics} синими линиями необходимо усреднить по числу электронов в пучке, усреднение по большому числу электронов даст спектр, который будет являть простой суммой $\sinc^2$ до каких либо дополнительных фаз, результат усреднения, приведёт к чёрной линии на рис.~\ref{fig:section_und_analitics}. Теперь здесь надо визуально показать процесс усреднения и заключить, что именно такой спектр будет наблюдаться на источнике.

\subsection{Учёт конечности эмиттанса}
В этом параграфе мы покажем влияние эмиттанса пучка на спектр излучения и угловое распределение. Для начала перепишем уравнение~\ref{eq:field_dist_Norm} с учётом отклонения частиц от заданной траектории, --- $h_x$ и $h_y$ и с некоторым дополнительным углом $\eta_x$ и $\eta_x$. Сразу можно понять, что в уравнениие~\ref{eq:field_dist_Norm} можно сделать замену $\theta_{x,y} \rightarrow \theta_{x,y} - \eta_{x, y} - \cfrac{l_{x,y}}{z_0}$ и переписать углы в нормализованных единицах аналогично с~\ref{eq:norm_units}, с точностью до фазы: 
\begin{equation}
	\label{eq:field_dist_Norm}
	\begin{array}{lcl}
	\hat{E}_{\bot} \sim
	\sinc\bigg[\cfrac{\hat{C}}{2} + 
	\cfrac{1}{4}\bigg(\hat{\theta}_{x} - \hat{\eta}_{x} - \cfrac{l_{x}}{z_0}\bigg)^2 +
	\cfrac{1}{4}\bigg(\hat{\theta}_{y} - \hat{\eta}_{y} - \cfrac{l_{y}}{z_0}\bigg)^2\bigg],
	\end{array}	
\end{equation}
При этом можно положить  $\cfrac{l_{x,y}}{z_0} \ll 1$, что выполняется с очень высокой точностью. 

В наших рассуждениях мы будем использовать один предельный случай: электронный пучок не симметричен его вертикальны размер много меньше размера по радиальному направлению. Распределение частиц будем считать гауссовым: 
\begin{equation}
	\label{eq:field_dist_Norm}
	h_{x, y}(\eta_{x, y}) = \cfrac{N_e}{\sqrt{2\pi}\sigma_{x', y'}} \exp[-\cfrac{\eta^2_{x, y}}{2\sigma^2_{x', y'}}]
\end{equation}

Для удобства перепишем это распределение в нормализованных единицах, помня $\sigma_{x', y'} = \epsilon_{x', y'}/\beta_{x', y'}$, где $\epsilon_{x', y'}$ --- вертикальный и горизонтальный эмиттансы, $\beta_{0x', y'}$ --- минимум бета-функции, обычно минимум бета-функции выбирают в середине ондулятора. Нормализованные единицы для $\hat{\beta}_{0} = \beta_{0}$
и $\hat{\epsilon} = (\omega/c)\epsilon$
\begin{equation}
	\label{eq:field_dist_Norm}
	h(\hat{\eta}) = \cfrac{1}{\sqrt{2\pi\hat{\epsilon}/\hat{\beta}}} \exp[-\cfrac{\hat{\eta}^2\hat{\beta}_{0}}{2\hat{\epsilon}^2}]
\end{equation}
Как уже упоминалось мы будем рассматривать предельный случай $\epsilon_{y'}/\beta_{y'} \ll 1$, в то время как $\hat{\beta_0}_{x,y} \sim 1$, поэтому просто $\epsilon_{y'} \ll 1$. Теперь можно записать интенсивность поля следующий образом: 
\begin{equation}
	\label{eq:convolution}
	\hat{I} = \cfrac{1}{\sqrt{2\pi\hat{\epsilon}/\hat{\beta}}}
	\displaystyle\int\limits_{-\infty}^{\infty} d\hat{\eta}_x \sinc^2(\zeta)	
	\exp[-\cfrac{\hat{\eta}_x^2\hat{\beta}_{0x}}{2\hat{\epsilon}_x}],д
\end{equation}
Где мы ввели $\zeta = \cfrac{\hat{C}}{2} + 
\cfrac{1}{4}(\hat{\theta}_{x} - \hat{\eta}_{x})^2 +
\cfrac{1}{4}\hat{\theta}_{y}^2$. Здесь мы учли, что распределение по $y$ действует как дельта-функция. Предыдущее уравнение упрощается дальше в пределе $\hat{\epsilon}_x\hat{\beta}_x \gg 1$, опять же помня, что и $\hat{\beta}_x \sim 1$, получается $\hat{\epsilon}_x \gg 1$. Ширина $\sinc^2(\zeta)$ много больше ширины гауссвоского распределения, ширина которого $\hat{\epsilon}_x$, поэтому интеграл будет набираться в пике кардинального синуса и экспоненту можно вынести с аргументом: $\hat{\eta}_x = \hat{\theta}_x$: 
\begin{equation}
	\label{eq:convolution}
	\hat{I} = \cfrac{\exp[-\hat{\theta}_x^2\hat{\beta}_{0x}/2\hat{\epsilon}_x]}{\sqrt{2\pi\hat{\epsilon}/\hat{\beta}}}
	\displaystyle\int\limits_{-\infty}^{\infty} d\hat{\eta}_x \sinc^2\bigg(\cfrac{\hat{C}}{2} + 
	\cfrac{1}{4}(\hat{\theta}_{x} - \hat{\eta}_{x})^2 +
	\cfrac{1}{4}\hat{\theta}_{y}^2\bigg)	
\end{equation}
Этот интеграл можно взять числено. %введя очевидную замену $\hat{\eta_{x}} \rightarrow \hat{\theta}_x - \hat{\eta_{x}}$. 
На~\ref{fig:2spec_emittance_and_single} представлены: линия 1.: спектр излучения пучка с $\hat{\epsilon}_x\rightarrow \infty$  $\hat{\epsilon}_x\rightarrow 0$, линия 2.: спектр одиночного электрона как функция $\hat{C} + \hat{\theta}_y^2/2$ при $\hat{\theta}_x = 0$, на рис.~\ref{fig:spec} тоже для распределения интенсивности по углам. 
\begin{figure}
	\centering  
	\begin{minipage}{0.49\textwidth}
		\centering
		\includegraphics[width=\textwidth]{pic/spec_integ_emittance.pdf}
		\caption{}
		\label{fig:2spec_emittance_and_single}
	\end{minipage}\hfill
	\begin{minipage}{0.49\textwidth}
		\centering
		\includegraphics[width=\textwidth]{pic/angle_integ_emittance.pdf}
		\caption{}
		\label{fig:spec}
	\end{minipage}    
\end{figure}
\section{Фурье оптика}
В этой главе мы предложим наглядный подход к решению задачи о распространение волнового фронта в пустом пространстве, его прохождении через систему линзу. Приведённые результаты напрямую могут быть использованы в программном коде. Распределение поля в начальный момент времени будем считать гауссовским, однако, как будет видно из изложения, подход может быть использован для произвольного распределения поля. В наших выкладкам мы в полной мере следуем подходу (Салдин, Serkez), более детальное описание можно найти в (Гудман)

\subsection{Распространение света в пустом пространстве}
Наши рассуждения мы начнём с волнового уравнения в пустом пространстве ($\vec{j} = 0, \rho = 0$). 
\begin{equation}
	\pdv[2]{\vec{E}}{t} + c^2 \nabla^2 \vec{E} = 0
\end{equation}

В  $r\omega$-пространстве уравнение приобретает знакомый вид уравнения Гельмгольца, где $k_0 = \omega/c$.

\begin{equation}
	k_0^2\vec{\widetilde{E}} + \nabla^2 \vec{\widetilde{E}} = 0
\end{equation}

Совершив фурье-преобразование в $k$-пространство по координатам $x,y$, которое определим схожим образом с~\ref{eq:Fourier_wt}:

\begin{equation}
	\label{eq:Fourier_rk}
	\begin{array}{lcl}
		\vec{\widehat{E}}(\vec{k}, \omega) = \displaystyle\int\limits_{-\infty}^{\infty}\int\limits_{-\infty}^{\infty} dxdy \vec{E}(\vec{r}, t)\exp[ik_xx + ik_xx]\\
		\\
		\vec{E}(\vec{r}, \omega) = \cfrac{1}{4\pi^2}\displaystyle\int\limits_{-\infty}^{\infty}\int\limits_{-\infty}^{\infty} dk_xdk_y \vec{\widehat{E}}(\vec{k}, t)\exp[-ik_xx - ik_xx],
	\end{array}
\end{equation}

получим: 
\begin{equation}
	k_0^2\Big(1 - \cfrac{k^2_x}{k^2_0} - \cfrac{k^2_y}{k^2_0} \Big)\vec{\widehat{E}} + \dv[2]{\vec{\widehat{E}}}{z} = 0
\end{equation}

Теперь можно напрямую можно получить решение этого обыкновенного дифференциального уравнения:
\begin{equation}
	\vec{\widehat{E}}(\omega, k_x, k_y, z) = \vec{\widehat{E}}(\omega, k_x, k_y, 0)\exp[ik_0z\sqrt{1 - \frac{k^2_x}{k^2_0} - \frac{k^2_y}{k^2_0}} ]
\end{equation}

Введём функцию отклика среды:

\begin{equation}
	\begin{array}{lcl}
	H(k_x, k_y, z) = \cfrac{\vec{\widehat{E}}(\omega, k_x, k_y, z)}{\vec{\widehat{E}}(\omega, k_x, k_y, 0)} = \exp[ik_0z\sqrt{1 - \cfrac{k^2_x}{k^2_0} - \cfrac{k^2_y}{k^2_0}} ]
	\\
	\\
	H(k_x, k_y, z) \approxeq \exp[k_0z]\exp[-\cfrac{iz}{2k_0}(k^2_x + k^2_y)]
	\end{array}
\end{equation}
 
Видно, чтобы получить распределение электромагнитного поля на некотором расстоянии $z$, необходимо совершить обратное преобразование Фурье в $xy$-пространство. Таким образом решение волнового уравнения сводиться к трём относительно простым операциям: первое, --- перевод начального распределения в $k_xk_y$-пространство, далее домножение получившегося распределения на функцию отклика среды, в нашем случае пустое пространство, и последний шаг, --- обратное преобразование Фурье. 

\subsection{Действие тонкой линзы на волновой фронт}
В этом параграфе мы построим элементарную оптическую систему, состоящую из пустого промежутка, - $d_1$, тонкой линзы с оптической силой, - $1/f$ и ещё одного пустого промежутка до плоскости изображения. Действие тонкой линзы мы представим как прибавление к фазе волны следующего выражения: 
\begin{equation}
	T_f(x, y) = \exp[-\cfrac{ik_0}{2f}(x^2 + y^2)]
\end{equation}

Для предметности обсуждения определим Гауссов пучок: 

\begin{equation}
	\overline{E}(x, y, 0) = A\exp[-\cfrac{x^2 + y^2}{w^2_0}]
\end{equation}

После Фурье преобразования в $k_xk_y$-пространстве мы получим:

\begin{equation}
	\hat{E}(k_x, k_y, 0) = A\pi w^2_0\exp[-\cfrac{w^2_0}{4}(k_x^2 + k_y^2)]
\end{equation}

После домножения этого распределения поля в $k_xk_y$-пространстве на функцию отклика пустого промежутка, получим
\begin{equation}
	\label{eq:after_lens}
	\begin{array}{lcl}
	\hat{E}(k_x, k_y, z) = \hat{E}(k_x, k_y, z)H(k_x, k_y, z) 
	\\
	\\
	= A\pi w^2_0\exp[-\cfrac{w^2_0}{4}(k_x^2 + k_y^2)]\exp[k_0z]\exp[-\cfrac{iz}{2k_0}(k^2_x + k^2_y)]
	\\
	\\
	= A\pi w^2_0\exp[k_0z]\exp[-\cfrac{iq}{2k_0}(k^2_x + k^2_y)], 
	\end{array}
\end{equation}
здесь мы ввели $q = z - iz_R$, где $z_R = \cfrac{k_0w^2_0}{2}$. После перехода обратно в $xy$-пространство, получим: 
\begin{equation}
	\overline{E}(x, y, z) = \cfrac{iAk_0w^2_0}{2q}\exp[k_0z]\exp[-i\cfrac{k_0}{2q}(x^2 + y^2)],
\end{equation}
Теперь можно воспользоваться выражением для тонкой линзы и получить:
\begin{equation}
	\begin{array}{lcl}
	\overline{E}_{l}(x, y, z) = T_f(x, y)\overline{E}(x, y, z) = 
	\\
	\\
	\cfrac{iAk_0w^2_0}{2q}\exp[k_0z]\exp[-i\cfrac{k_0}{2q}(x^2 + y^2)]\exp[-\cfrac{ik_0}{2f}(x^2 + y^2)]
	\\
	\\
	\cfrac{iAk_0w^2_0}{2q}\exp[k_0z]\exp[-i\cfrac{k_0}{2q_l}(x^2 + y^2)],
	\end{array}
\end{equation}
где $\cfrac{1}{q_l} = \cfrac{1}{q} \; - \; \cfrac{1}{f}$. Теперь можно подвести итог: после распространения волнового фронта на расстояние $d_1$ параметр $q$ преобразуется:
\begin{equation}
	q(d_1) = q(0) + d_1, 
\end{equation}
далее на него действует линза: 
\begin{equation}
	\cfrac{1}{q_l} = \cfrac{1}{q(0) + d_1} - \cfrac{1}{f},
\end{equation}
и ещё один пустой промежуток, до места, где волновой фронт опять будет плоским: 
\begin{equation}
	q(d_1 + d_2) = q_l + d_2, 
\end{equation}
Условие того, что волновой фронт плоский мы сформулируем так, что $q(d_1 + d_2) = -i\cfrac{k_0w^2_2}{2}$, что легко проверятся подстановкой в~\ref{eq:after_lens}. Получим уравнение: 
\begin{equation}
	-i\cfrac{k_0w^2_2}{2} = q_l + d_2, 
\end{equation}
где, приравниванием мнимых частей получим: 
\begin{equation}
	w^2_2 = \cfrac{f^2w^2_1}{(f - d_1)^2 + (k_0w^2_1/2)^2}
\end{equation}
то же для реальных частей: 
\begin{equation}
	d_2 = f + f^2\cfrac{(d_1 - f)}{(d_1 - f)^2 + (k_0w^2_1/2)^2}
\end{equation}
Из последнего уравнения видно, что если положить перетяжку гауссового пучка раной нулю, то выражение переходит в обычное соотношение геометрической оптики.

В приведённой главе мы дали краткий путь того, как можно очень эффективно и относительно просто использовать Фурье оптику для написания симуляционных кодов при проектировании оптических систем. В качестве примера, для заинтересованных читателей на веб-странице (веб-страница) приведёт код простой оптической системы, который в полной мере используют результаты вышеприведённого параграфа. Код был написан автором данной рукописи рамках курса <<Основы вычислительной физики>>, который читается на физическом факультете НГУ. Дальнейшие комментарии к коду можно найти в репозитории указанной по ссылке.
\section{Краткий обзор дифракции на кристаллах}
В это главе мы кратко дадим основные результаты кинетической и динамической теории дифракции. Основные кристаллы используемы на синхротронных источниках третьего и четвёртого поколений - это $Si$ (кремний), $C$ (алмаз) и реже $Ge$ (германий), в виду свой кубической кристаллический решётки, эти кристаллы относительно просты для анализа. Для нас важны такие свойства кристаллов, как способность преобразовать относительно широкой спектр излучения ондулятора, в излучение с относительной монохроматичностью до $\Delta E/ E \sim 10^{-4}$, а также поглощательные способности кристаллов, что в значительной степени снижает тепловые нагрузки на оптические элементы.
\subsection{Симметричное брэгговское отражение от идеально кристалла}
Длины волн, которые отвечают резонансу при отражении падающего под углом $\theta$ к плоскости кристалла излучения, даётся законом Брэгга:  
%пропорциональный $\gamma / nN$, где $n$ --- номер гармоники излучения, $N$ --- число периодов ондулятора, а $\gamma$ --- гамма фактор релятивистского электрона
\begin{equation}
	m\lambda = 2d\sin\theta,
\end{equation}
где $d$ --- расстояние между плоскостями от которых происходит отражение, а $m$ --- некоторое положительно целое число. Основной результат, который мы будем использовать, это кривая Дарвина, которая определяет угловой акцептанс излучения. Динамическая и кинематическая теории дифракции дают конечную ширину в, которую кристалл может принять излучение, а также некоторый сдвиг, относительно предполагаемого брэгговского угла. На рис.~\ref{fig:bragg_R} показаны характерные кривые отражение. По ним видно, что чем больше энергия подающего пучка излучения, тем уже кривая и ближе к даваемому законом брэгга углу. При расчёте кристаллов монохроматоров этот факт необходимо учитывать, так как излучение не попавшее в акцептанс кристалла будет поглощаться и выделять в нем тепло. 
\subsection{Поглощательные способности кристаллов}
Одним из полезных применение кристаллов в рентгеновском диапазоне есть их фильтрующая способность, отрезать низкие энергии, в особенности для алмазных кристаллов, которые, по мимо всего, имеют хорошую теплопроводность, что способствуют быстрому теплоотводу. На рис.~\ref{fig:bragg_T} представлена кривая поглощения 100 мкм кристалла алмаза. Подобные кристаллы устанавливают перед первыми оптическими элементами, что в значительной степени снижает тепловые нагрузки, подавлением низших гармоник.
\begin{figure}
	\centering  
	\begin{minipage}{0.49\textwidth}
		\centering
		\includegraphics[width=\textwidth]{pic/bragg_R.pdf}
		\caption{}
		\label{fig:bragg_R}
	\end{minipage}\hfill
	\begin{minipage}{0.49\textwidth}
		\centering
		\includegraphics[width=\textwidth]{pic/bragg_T.pdf}
		\caption{}
		\label{fig:bragg_T}
	\end{minipage}    
\end{figure}






















