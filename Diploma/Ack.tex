\chapter*{Благодарности}
Автор рукописи считает своим долгом выразить благодарности людям, которые способствовали написанию этого труда и поддерживали его на пути получения степени бакалавра в Новосибирском Государственном Университете по специальности физика.\\

Во-первых, выражаю благодарность моему научному руководителю \textbf{Якову Валерьевичу Ракшуну} за возможность работать в передовом проекте СО РАН, поддержке моих инициатив и подаче бесценных советов по работе в проектном офисе ЦКП <<СКИФ>>

Не могу не высказать глубокую признательность моим наставникам: \textbf{Евгению Салдину} за его терпеливые ответы на мои вопросы по теории синхротронного излучения и чуткие наставления в выборе моей специализации, \textbf{Dr Svitozar Serkez} и \textbf{Dr Gianluca Geloni} за бесценный вклад в приобретении мной всех необходимых навыков работы и возможности стажироваться в их исследовательской группе на European XFEL.

\textbf{Тельнову Валерию Ивановичу}, \textbf{Никитину Сергею Алексеевичу} и \textbf{Никитиной Людмиле Константиновне} за помощь в переводе на Кафедру Ускорителей, их бесценные советы и поддержку.

Хочу выразить благодарность всему преподавательскому составу Новосибирского Государственного Университета и отдельно преподавателями Кафедры Ускорителей за их нелёгкий труд.

В заключении, выражаю благодарность моим родителям: \textbf{Евгению Требушинину} и \textbf{Татьяне Требушининой} за их поддержку и вдохновение на упорный труд, а так же \textbf{Александре Голубевой}.