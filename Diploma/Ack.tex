\chapter*{Благодарности}
\addcontentsline{toc}{chapter}{Благодарности}	% Добавляем его в оглавление
Автор работы считает приятным долгом выразить слова благодарности людям, которые способствовали написанию этой работы и поддерживали его на пути получения степени бакалавра в Новосибирском государственном университете по направлению физика.

Во-первых, благодарность научному руководителю работы, \textbf{Ракшуну Якову Валерьевичу}, за возможность работать в передовом проекте СО РАН и поддержке инициатив автора данной работы.

Слова благодарности рецензенту работы, \textbf{Ращенко Сергею Владимировичу}, за конструктивные предложения по улучшению содержания работы и стиля текста.

Благодарность всему преподавательскому составу Новосибирского государственного университета и отдельно преподавателями кафедры физики ускорителей за их нелёгкий труд.

Глубокая признательность наставникам автора: \textbf{Евгению Салдину} за его терпеливые ответы на вопросы по теории синхротронного излучения и чуткие наставления в выборе специализации, \textbf{Cвiтозару Серкезу} и \textbf{Джанлуке Гелони} за бесценный вклад в приобретении всех необходимых навыков работы и возможности стажироваться в их исследовательской группе на European XFEL.

Слова признательности \textbf{Тельнову Валерию Ивановичу}, \textbf{Никитину Сергею Алексеевичу} и \textbf{Никитиной Людмиле Константиновне} за помощь в переводе на кафедру физики ускорителей, их советы и внимание.

В заключении, благодарность \textbf{Евгению Требушинину} и \textbf{Татьяне Требушининой} за их терпение и поддержку, \textbf{Александре Голубевой} за вдохновение на упорный труд.