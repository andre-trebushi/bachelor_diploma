\chapter{Элементы фурье оптики}
В этой главе мы предложим наглядный подход к решению задачи о распространение волнового фронта в пустом пространстве, его прохождении через систему линзу. Приведённые результаты напрямую могут быть использованы в программном коде. Распределение поля в начальный момент времени будем считать гауссовским, однако, как будет видно из изложения, подход может быть использован для произвольного распределения поля. В наших выкладкам мы в полной мере следуем подходу \cite{serkez2013grating} и \cite{serkez2015design}, полное изложение фурье оптики и статистической оптики можно найти в замечательных книгах \cite{goodman2015statistical}, \cite{goodman2005introduction}. В конце главы будет приведёт пример учебного кода \cite{my_github} для распространения волнового фронта через оптическую систему, написанный автором в рамках одного из университетских курсов.

\section{Распространение света в пустом пространстве}
Наши рассуждения мы начнём с волнового уравнения в пустом пространстве, т.е. $\vec{j} = 0, \rho = 0$. 
\begin{equation}
	\pdv[2]{\vec{E}}{t} + c^2 \nabla^2 \vec{E} = 0
\end{equation}
В  $r\omega$-пространстве уравнение приобретает знакомый вид уравнения Гельмгольца, где $k_0 = \omega/c$.

\begin{equation}
	k_0^2\vec{\widetilde{E}} + \nabla^2 \vec{\widetilde{E}} = 0
\end{equation}
Совершив фурье-преобразование в $k$-пространство по координатам $x,y$, которое определим схожим образом с~\ref{eq:Fourier_wt}:

\begin{equation}
	\label{eq:Fourier_rk}
	\begin{array}{lcl}
		\vec{\widehat{E}}(\vec{k}, \omega) = \displaystyle\int\limits_{-\infty}^{\infty}\int\limits_{-\infty}^{\infty} dxdy \vec{E}(\vec{r}, t)\exp[ik_xx + ik_xx]\\
		\\
		\vec{E}(\vec{r}, \omega) = \cfrac{1}{4\pi^2}\displaystyle\int\limits_{-\infty}^{\infty}\int\limits_{-\infty}^{\infty} dk_xdk_y \vec{\widehat{E}}(\vec{k}, t)\exp[-ik_xx - ik_xx],
	\end{array}
\end{equation}
получим: 
\begin{equation}
	k_0^2\Big(1 - \cfrac{k^2_x}{k^2_0} - \cfrac{k^2_y}{k^2_0} \Big)\vec{\widehat{E}} + \dv[2]{\vec{\widehat{E}}}{z} = 0
\end{equation}
Теперь можно напрямую можно получить решение этого обыкновенного дифференциального уравнения:
\begin{equation}
	\label{eq:sol}
	\vec{\widehat{E}}(\omega, k_x, k_y, z) = \vec{\widehat{E}}(\omega, k_x, k_y, 0)\exp[ik_0z\sqrt{1 - \frac{k^2_x}{k^2_0} - \frac{k^2_y}{k^2_0}} ]
\end{equation}
На основе уравнения~\ref{eq:sol} введём функцию отклика среды:

\begin{equation}
	\begin{array}{lcl}
	H(k_x, k_y, z) = \cfrac{\vec{\widehat{E}}(\omega, k_x, k_y, z)}{\vec{\widehat{E}}(\omega, k_x, k_y, 0)} = \exp[ik_0z\sqrt{1 - \cfrac{k^2_x}{k^2_0} - \cfrac{k^2_y}{k^2_0}} ]
	\\
	\\
	H(k_x, k_y, z) \approxeq \exp[k_0z]\exp[-\cfrac{iz}{2k_0}(k^2_x + k^2_y)]
	\end{array}
\end{equation}
Видно, чтобы получить распределение электромагнитного поля на некотором расстоянии $z$, необходимо совершить обратное преобразование Фурье в $xy$-пространство. Таким образом решение волнового уравнения сводиться к трём относительно простым операциям: первое, --- перевод начального распределения в $k_xk_y$-пространство, далее домножение получившегося распределения на функцию отклика среды, в нашем случае пустое пространство, и последний шаг, --- обратное преобразование Фурье. Из вывода видно,что мы не накладывали никаких ограничений на начальное распределение поля, кроме, быть может, естественных ограничений, накладываемых на оригинал преобразованием Фурье.

\section{Действие тонкой линзы на волновой фронт}
В этом параграфе мы построим элементарную оптическую систему, состоящую из пустого промежутка, --- $d_1$, тонкой линзы с оптической силой, --- $1/f$ и ещё одного пустого промежутка до плоскости изображения. Действие тонкой линзы мы представим как домножение комплексной амплитуды поля на следующее выражение: 
\begin{equation}
	T_f(x, y) = \exp[-\cfrac{ik_0}{2f}(x^2 + y^2)]
\end{equation}
Для предметности обсуждения определим начальное распределение гауссовым пучком: 

\begin{equation}
	\overline{E}(x, y, 0) = A\exp[-\cfrac{x^2 + y^2}{w^2_0}]
\end{equation}
После преобразование Фурье в $k_xk_y$-пространстве мы получим:

\begin{equation}
	\hat{E}(k_x, k_y, 0) = A\pi w^2_0\exp[-\cfrac{w^2_0}{4}(k_x^2 + k_y^2)]
\end{equation}
После домножения этого распределения поля в $k_xk_y$-пространстве на функцию отклика пустого промежутка, получим
\begin{equation}
	\label{eq:after_lens}
	\begin{array}{lcl}
	\hat{E}(k_x, k_y, z) = \hat{E}(k_x, k_y, z)H(k_x, k_y, z) 
	\\
	\\
	= A\pi w^2_0\exp[-\cfrac{w^2_0}{4}(k_x^2 + k_y^2)]\exp[k_0z]\exp[-\cfrac{iz}{2k_0}(k^2_x + k^2_y)]
	\\
	\\
	= A\pi w^2_0\exp[k_0z]\exp[-\cfrac{iq}{2k_0}(k^2_x + k^2_y)], 
	\end{array}
\end{equation}
здесь мы ввели $q = z - iz_R$, где $z_R = \cfrac{k_0w^2_0}{2}$. После перехода обратно в $xy$-пространство, получим: 
\begin{equation}
	\overline{E}(x, y, z) = \cfrac{iAk_0w^2_0}{2q}\exp[k_0z]\exp[-i\cfrac{k_0}{2q}(x^2 + y^2)],
\end{equation}
Теперь можно воспользоваться выражением для тонкой линзы и получить:
\begin{equation}
	\begin{array}{lcl}
	\overline{E}_{l}(x, y, z) = T_f(x, y)\overline{E}(x, y, z) = 
	\\
	\\
	\cfrac{iAk_0w^2_0}{2q}\exp[k_0z]\exp[-i\cfrac{k_0}{2q}(x^2 + y^2)]\exp[-\cfrac{ik_0}{2f}(x^2 + y^2)]
	\\
	\\
	\cfrac{iAk_0w^2_0}{2q}\exp[k_0z]\exp[-i\cfrac{k_0}{2q_l}(x^2 + y^2)],
	\end{array}
\end{equation}
где $\cfrac{1}{q_l} = \cfrac{1}{q} \; - \; \cfrac{1}{f}$. \\

Теперь можно подвести итог: после распространения волнового фронта на расстояние $d_1$ параметр $q$ преобразуется:
\begin{equation}
	q(d_1) = q(0) + d_1, 
\end{equation}
далее на него действует линза: 
\begin{equation}
	\cfrac{1}{q_l} = \cfrac{1}{q(0) + d_1} - \cfrac{1}{f},
\end{equation}
и ещё один пустой промежуток, до места, где волновой фронт опять будет плоским: 
\begin{equation}
	q(d_1 + d_2) = q_l + d_2, 
\end{equation}
Условие того, что волновой фронт плоский мы сформулируем так, что $q(d_1 + d_2) = -i\cfrac{k_0w^2_2}{2}$, что легко проверятся подстановкой в~\ref{eq:after_lens}. Получим уравнение: 
\begin{equation}
	-i\cfrac{k_0w^2_2}{2} = q_l + d_2, 
\end{equation}
где, приравниванием мнимых частей, получим: 
\begin{equation}
	w^2_2 = \cfrac{f^2w^2_1}{(f - d_1)^2 + (k_0w^2_1/2)^2}
\end{equation}
то же для реальных частей: 
\begin{equation}
	d_2 = f + f^2\cfrac{(d_1 - f)}{(d_1 - f)^2 + (k_0w^2_1/2)^2}
\end{equation}
Из последнего уравнения видно, что если положить перетяжку гауссового пучка раной нулю, то выражение переходит в соотношение геометрической оптики.

В приведённой главе мы дали краткий путь того, как можно очень эффективно и относительно просто использовать Фурье оптику для написания симуляционных кодов при проектировании оптических систем. В качестве примера, для заинтересованных читателей на веб-странице (веб-страница) приведёт код простой оптической системы, который в полной мере используют результаты вышеприведённого параграфа. Код был написан автором данной рукописи рамках курса <<Основы вычислительной физики>>, который читается на физическом факультете НГУ. Дальнейшие комментарии к коду можно найти в репозитории указанной по ссылке.
