\chapter{Ондуляторное излучение}
В этой главе будет дан вывод излучения релятивистского электрона в $r\omega$-пространстве, движущегося в синусоидальном магнитном поле. Вывод замечателен тем, что даёт результаты из первых принципов --- уравнений Максвелла, а точность используемых приближений можно наглядно проследить по ходу изложения. Приведённые выкладки повторяют подход разработанный в серии работ \cite{geloni2005paraxial} -
%, \cite{geloni2007fourier}, \cite{geloni2015brightness },
 \cite{geloni2006fourier}. В заключении главы будет дан обзор на код SRW --- Synchrotron Radiation Workshop \cite{SRW} - \cite{chubar1998proceedings}, а также даны краткие описания других симуляционных кодов, которые активно используются в научном сообществе для расчёта синхротронного излечения. 
\section{Излучение релятивистского электрона в синусоидальном магнитном поле}
\subsection{Уравнение движения электрона в ондуляторе}
Вывод спектра излучения ондулятора начнётся с уравнения движение релятивистского электрона в магнитном поле:
\begin{equation}
	\vec{F} = e[\vec{v} \times \vec{B}],
\end{equation} 
где $e$ --- заряд электрона, а $\vec{v}$ и $\vec{B}$ --- скорость частицы и магнитное поле, соответственно. Уравнение можно переписать в виде:

\begin{equation}
	\label{eq:NewTown}
	\cfrac{d\vec{p}}{dt} = \cfrac{e}{\gamma m_e}[\vec{v} \times \vec{B}],
\end{equation}
где $\gamma$ --- лоренц фактор, появившийся из релятивистского импульса. Отложим ось $z$ вдоль направления релятивистского движения электрона и будем считать, магнитное поле в ондуляторе $B_0\cos(k_w z)$ направлено вдоль оси $y$, где $k_w$ связана с периодом ондулятора следующим образом $k_w = 2\pi/\lambda_w$. После этого уравнение~\ref{eq:NewTown} можно переписать в виде:

\begin{equation}
	\label{eq:eq_of_motion}
	\begin{cases}
		\cfrac{d^2 x}{dt^2} = - \cfrac{e B_0}{\gamma m_e}\cfrac{dz}{dt} \cos(k_w z)\\
		\cfrac{d^2 z}{dt^2} = \cfrac{e B_0}{\gamma m_e}\cfrac{dx}{dt} \cos(k_w z)
	\end{cases} 
\end{equation}
далее, один раз интегрируя первое уравнение системы с заменой $dz = \beta cdt$, где $\beta = \|\vec{v}\| /c$, можно получить: 

\begin{equation}
 	\label{eq:dx/dt}
	\cfrac{dx}{dt} = - \cfrac{eB_0}{\gamma m_ek_w} \sin(k_w z)
\end{equation}
Введём коэффициент ондуляторности --- $K = \cfrac{eB_0 \lambda_u}{2 \pi m_e c}$, который показывает угол отклонения траектории электрона от оси $z$. 

Подставляя получившийся результат~\ref{eq:dx/dt} во второе уравнение системы~\ref{eq:eq_of_motion} и интегрируя с пределами от $0$ до некоторого $z_0$, получим систему:

\begin{equation}
	\begin{cases}
	\label{eq:eq_of_motion_velocity}
		\cfrac{dx}{dt} = - \cfrac{Kc}{\gamma} \sin(k_w z)\\
		\cfrac{dz}{dt} = \beta c - \cfrac{K^2 c}{2 \gamma^2 \beta}\sin^2(k_w z),
	\end{cases} 
\end{equation}
чтобы получить уравнение на траекторию частицы, ещё раз проинтегрируем оба уравнения и, в итоге, получим:

\begin{equation}
	\begin{cases}
	\label{eq:eq_of_motion_trej}
		x = \cfrac{Kc}{\gamma k_w \beta} \cos(k_w\overline{\beta}ct)\\
		z = \overline{\beta}ct + \cfrac{K^2}{8 \beta^2 \gamma^2 k_w}\sin(2k_w\overline{\beta}ct) 
	\end{cases} 
\end{equation}
Здесь введено обозначение $\overline{\beta}$, которое определяется следующим образом $\overline{\beta}c = \beta c\bigg(1 - \cfrac{K^2}{4 \beta^2 \gamma^2}\bigg)$. Полученные решения будут использоваться при интегрировании уравнений Максвелла.
\subsection{Решение уравнений Максвелла в параксиальном приближении}
Вывод спектра излучения будем проводить в $r\omega$-пространстве. Начнём с уравнений Максвелла в вакууме:
\begin{equation}
	\begin{cases}
		\nabla \cdot \vec{E} = 4\pi \rho\\
		\nabla \cdot \vec{B} = 0\\
		[\nabla \times \vec{E}] = -\cfrac{1}{c} \cfrac{\partial\vec{B}}{\partial t}\\
		[\nabla \times \vec{B}] = \cfrac{4\pi}{c} \vec{j} + \cfrac{1}{c} \cfrac{\partial\vec{E}}{\partial t}.
	\end{cases} 
\end{equation}
Из уравнений тривиально можно получить неоднородное волновое уравнение: 
\begin{equation}
	\label{eq:inhomo_wave_eq_xt}
	c^2 \nabla^2 \vec{E} - \pdv[2]{\vec{E}}{t} = 4\pi c^2 \nabla \rho + 4\pi \pdv{\vec{j}}{t}.
\end{equation}
Это же уравнение перепишем в $r\omega$-пространстве, определив преобразование Фурье следующим образом:
\begin{equation}
	\label{eq:Fourier_wt}
	\begin{array}{lcl}
		\vec{\widetilde{E}}(r, \omega) = \displaystyle\int\limits_{-\infty}^{\infty} dt \vec{E}(r, t)\exp[i\omega t]\\
		\\
		\vec{E}(r, \omega) = \cfrac{1}{2\pi}\displaystyle\int\limits_{-\infty}^{\infty} d\omega \vec{\widetilde{E}}(r, t)\exp[-i\omega t]
	\end{array}
\end{equation}
Применив к уравнению~\ref{eq:inhomo_wave_eq_xt}, получим:
\begin{equation}
	\label{eq:inhomo_wave_eq_xw}
	\omega^2 \vec{\widetilde{E}} + c^2 \nabla^2 \vec{\widetilde{E}} = 4\pi c^2 \nabla  \widetilde{\rho} - 4i\pi\omega\vec{\widetilde{j}}.
\end{equation}
Перепишем это уравнение в приближении медленно меняющейся амплитуды в сравнение с частотой осцилляций, что есть $\vec{\widetilde{E}} =  \vec{\overline{E}}\exp[i\omega z/c]$, в приближении $\cfrac{\partial |\vec{E}|}{\partial z} \ll \cfrac{\omega}{c}|\vec{E}|$, где временная зависимость разложена до нулевого порядка малости. Исходя из уравнения~\ref{eq:inhomo_wave_eq_xw}, получим:
\begin{equation}
	\label{eq:wave_slow_vary}
	c^2\bigg(\nabla^2 \vec{\widetilde{E}} + \cfrac{2i\omega}{c}\pdv{\vec{\widetilde{E}}}{z}\bigg)\exp[i\omega z/c] = 4\pi c^2 \nabla  \widetilde{\rho} - 4i\pi\omega\vec{\widetilde{j}}.
\end{equation}
Для электрона движущегося в вакууме ток и плотность заряда выражается через дельта-функцию Дирака:
\begin{equation}
	\begin{array}{lcl}
		\rho(r,t) = -e\delta(\vec{r}- \vec{r'}(t)) = -\cfrac{e}{v_z(z)}\delta(\vec{r}_{\bot}- \vec{r'}_{\bot}(z))\delta(\cfrac{s(z)}{v} - t)\\
		\vec{j}(r,t) = \vec{v}\rho(r,t),
	\end{array}
\end{equation} 
в $r\omega$-пространстве: 
\begin{equation}
	\begin{array}{lcl}
		\widetilde{\rho}(r,\omega) = -\cfrac{e}{v_z(z)}\delta(\vec{r}_{\bot}- \vec{r'}_{\bot}(z))\exp[\cfrac{iws(z)}{v}]\\
		\widetilde{\vec{j}}(r,\omega) = \vec{v}\widetilde{\rho}(r,\omega).	
	\end{array}
\end{equation} 
Подставим Фурье-образы плотности тока и заряда в уравнение~\ref{eq:wave_slow_vary}:
\begin{equation}
	\label{eq:wave_eq}
	\begin{array}{lcl}
		\nabla^2 \vec{\widetilde{E}} + \cfrac{2i\omega}{c}\cfrac{\partial\vec{\widetilde{E}}}{\partial z} = 
		\cfrac{4\pi e}{v_z(z)} \exp[iw\bigg(\cfrac{s(z)}{v} - \cfrac{z}{c}\bigg)]
		\bigg(  
			\cfrac{i\omega}{c^2}\vec{v}(z)
			-\nabla\bigg) \delta(\vec{r}_{\bot} - \vec{r'}_{\bot}(z)).
		
	\end{array}
\end{equation} 
Получившиеся уравнение является точным. Теперь можно применить параксиальное приближение. 
\begin{equation}
	\label{eq:wave_slow_vary_parax}
	\begin{array}{lcl}
		\nabla_{\bot}^2 \vec{\widetilde{E}}_{\bot} + \cfrac{2i\omega}{c}\cfrac{\partial\vec{\widetilde{E}}_{\bot}}{\partial z} = 
		\cfrac{4\pi e}{v_z(z)} \exp[iw\bigg(\cfrac{s(z)}{v} - \cfrac{z}{c}\bigg)]\bigg(  
			\cfrac{i\omega}{c^2}\vec{v}_{\bot}(z) 
			-\nabla_{\bot}\bigg) \delta(\vec{r}_{\bot} - \vec{r'}_{\bot}(z)).
	\end{array}
\end{equation} 
Перед нами неоднородное дифференциальное уравнение в частных производных, которое будет решено с помощью функции Грина. Для дифференциального оператора $\partial_t - k\nabla_{2D}^2$ функция Грина есть: $\cfrac{1}{4\pi kt}\exp[-\rho^2/4kt]$. В частности для уравнения~\ref{eq:wave_slow_vary_parax}
\begin{equation}
	\label{eq:Green_func}
	G(z_0 - z'; \vec{r}_{\bot 0} - \vec{r'}_{\bot}) = 
	- \cfrac{1}{4\pi (z_0 - z')}\exp[i\omega \cfrac{|\vec{r}_{\bot 0} - \vec{r'}_{\bot}|^2}{2c(z_0 - z')}].
\end{equation} 
Получим решение для распределения поля:

\begin{equation}
	\begin{array}{lcl}
		\vec{\widetilde{E}}_{\bot}(z_0,  \vec{r}_{\bot 0}, \omega) = -\cfrac{e}{c}  \displaystyle\int\limits_{-\infty}^{\infty}\int\limits_{-\infty}^{\infty} dz'd\vec{r'}\cfrac{1}{z_0 - z'}
		\bigg(\cfrac{i\omega}{c^2}\vec{v}_{\bot}(z')
		-\nabla'_{\bot}\bigg) \delta(\vec{r'}_{\bot} - \vec{r'}_{\bot}(z'))\times\\
		\exp[iw\bigg( \cfrac{|\vec{r}_{\bot 0} - \vec{r'}_{\bot}|^2}{2c(z_0 - z')} +\cfrac{s(z')}{v} - \cfrac{z'}{c} \bigg)].
	\end{array}	
\end{equation}
Проинтегрировав по $d\vec{r'}$, получим общее решение уравнения~\ref{eq:wave_eq} :
\begin{equation}
	\label{eq:field_in_parax_com}
	\begin{array}{lcl}
		\vec{\widetilde{E}}_{\bot}(z_0,  \vec{r}_{\bot 0}, \omega) = -\cfrac{i\omega e}{c^2}  \displaystyle\int\limits_{-\infty}^{\infty} dz'
		\cfrac{1}{z_0 - z'}
		\bigg(\cfrac{\vec{v}_{\bot}(z')}{c}
		- \cfrac{\vec{r}_{\bot 0} - \vec{r'}_{\bot}(z')}{(z_0 - z')}\bigg)\times\\
		\exp[iw\bigg(\cfrac{|\vec{r}_{\bot 0} - \vec{r'}_{\bot}(z')|^2}{2c(z_0 - z')} + \cfrac{s(z')}{v} - \cfrac{z'}{c} \bigg)].
	\end{array}	
\end{equation}
Итого, получено распределение электромагнитного поля в точке наблюдения $\vec{r}_0$, которое получит явный вид после интегрирования по траектории $\vec{r'}_{\bot}(z')$.

\subsection{Излучение планарного ондулятора}
В этой секции будет рассмотрено излучение планарного ондулятора, c использованием решения уравнений Максвелла~\ref{eq:field_in_parax_com} и траектории движения электрона в синусоидальном магнитном поле~\ref{eq:eq_of_motion_trej}. Сперва проанализируем получившиеся распределение поля~\ref{eq:field_in_parax}: в случае ондулятора, член $(z_0 - z')^{-1}$ можно разложить около $z'$, что всегда верно для дальней зоны, так как размер ондулятора много меньше расстояния, с которого наблюдается излучения: $\lambda_w N \ll z_0$, где $N$ число периодов ондулятора.

Воспользовавшись решениями~\ref{eq:eq_of_motion_velocity} и~\ref{eq:eq_of_motion_trej} и помня $\vec{r}_{\bot 0}/z_0 = \vec{\theta}$, преобразуем уравнение~\ref{eq:field_in_parax_com} к виду:
\begin{equation}
	\label{eq:field_in_parax}
	\begin{array}{lcl}
		\vec{\widetilde{E}}_{\bot}(z_0,  \vec{r}_{\bot 0}, \omega) =
		\cfrac{i\omega e}{c^2z_0} \exp[i\cfrac{\omega \theta^2 z_0}{2c}]
	 	\displaystyle\int\limits_{-\lambda_w N/2}^{\lambda_w N/2} dz'\exp[i\Phi_T]
		\bigg(\cfrac{K}{\gamma}\sin(k_w z)\vec{e}_x + \vec{\theta}\bigg)
	\end{array}	
\end{equation}
Здесь отброшены члены первого и больших порядков малости по $1/z_0$. За $\Phi_T$ обозначено следующее выражение:
\begin{equation}
	\Phi_T = 
	\bigg(\cfrac{\omega}{2c\widetilde{\gamma}^2} + 
	\cfrac{\omega\vec{\theta}^2}{2c}\bigg)z' - 
	\cfrac{K^2}{8\gamma^2}\cfrac{\omega}{k_w c}\sin(2k_wz') - \cfrac{K{\theta_x}}{\gamma}\cfrac{\omega}{k_w c}\cos(k_w z'),
\end{equation}
а $\widetilde{\gamma} = \cfrac{\gamma}{\sqrt{1 + K^2/2}}$.\\

Пределы интегрирования ограничены длиной ондулятора от $-\lambda_w N/2$ до $\lambda_w N/2$, считается, что вклад в излучение ондулятора является доминирующим над вкладами от остальных участков траектории. На этом шаге уже можно заметить, что излучение на оси будет линейно поляризованно. По ходу выкладок можно проследить, что это есть вклад токового члена из уравнения~\ref{eq:inhomo_wave_eq_xw}, вклад же плотности заряда или, далее называемый, градиентным членом, даёт вариацию поляризации при наблюдении под некоторым углом $\vec{\theta}$ к оси.
Перепишем~\ref{eq:field_in_parax} в следующе виде:
\begin{equation}
		\label{eq:field_in_parax_Bessel}
		\begin{array}{lcl}
			\vec{\widetilde{E}}_{\bot}(z_0,  \vec{r}_{\bot 0}, \omega) =
			\cfrac{i\omega e}{c^2z_0} \exp[i\cfrac{\omega \vec{\theta}^2 z_0}{2c}]
			\displaystyle\sum_{m,n=-\infty}^{+\infty}
			J_m\bigg(-\cfrac{K^2}{8\gamma^2}\cfrac{\omega}{k_w c}\bigg)
			J_n\bigg(-\cfrac{K{\theta_x}}{\gamma}\cfrac{\omega}{k_w c}\bigg)\times\\
			\exp[\cfrac{i\pi n}{2}]
			\displaystyle\int\limits_{-\lambda_w N/2}^{\lambda_w N/2} dz'\exp[i(2m + n)k_wz']
			\bigg(\cfrac{K}{2i\gamma}\big(\exp[2ik_w z'] - 1\big)\vec{e_x} + \vec{\theta}\exp[ik_w z']\bigg)\times\\
			\exp[i\bigg(k_w \cfrac{\Delta\omega}{\omega_r} + 
			\cfrac{\omega\vec{\theta}^2}{2c}\bigg)z'],
		\end{array}	
\end{equation}
Где введено $\omega = \omega_r + \Delta\omega$, $\omega_r = 2c\widetilde{\gamma}^2k_w$ и использовали формулу Якоби — Ангера:
\begin{equation}
	\begin{array}{lcl}
		\exp[iz\cos(\theta)] = 
		\displaystyle\sum\limits_{n =-\infty}^{\infty}
		i^n J_n(z)\exp[in\theta]\\	
		\exp[iz\sin(\theta)] = 
		\displaystyle\sum\limits_{n =-\infty}^{\infty}
		J_n(z)\exp[in\theta].
	\end{array}	
\end{equation}

До сих пор использовалось только одно приближение при решении уравнения Максвелла --- параксиальное приближение, теперь можно воспользоваться следующим параметром --- количеством периодов ондулятора $N$. Для этого обратим внимание на первое слагаемое в фазовом множителе под интегралом и заметим, что если $k_w \cfrac{\Delta\omega}{\omega_r} + 
\cfrac{\omega\vec{\theta}^2}{2c} \ll k_w$, то фаза меняется медленно на одном периоде и не занулит интеграл. Отметим, что для резонанса оба слагаемых должны быть много меньше единицы, т.е. $\Delta\omega/\omega_r \ll 1$ и $\omega\vec{\theta}^2/2c \ll 1$, последнее соотношение даёт углы наблюдения вблизи резонанса: $\theta \ll 1/\widetilde{\gamma}$. Теперь следует обратить внимание на аргументы функций Бесселя, а именно: 
\begin{equation}
	\begin{array}{lcl}
		u = -\cfrac{K^2}{8\gamma^2}\cfrac{\omega}{k_w c}\\
		v = -\cfrac{K{\theta_x}}{\gamma}\cfrac{\omega}{k_w c} = - \cfrac{K{\theta_x}}{\gamma}
		\bigg(1 + \cfrac{\Delta\omega}{\omega_r}\bigg)2\widetilde{\gamma}^2 \lesssim
		\cfrac{2K{\theta_x}\widetilde{\gamma}}{\sqrt{1 + K^2/2}} \lesssim \theta_x\widetilde{\gamma} \ll 1.
	\end{array}	
\end{equation}
Зная, что $J_\alpha(x) \thicksim \displaystyle\sum\limits_{n =0}^{\infty} x^{2n + \alpha} $, видим, что вклад нулевого порядка по $\theta_x\widetilde{\gamma}$, т.е. $J_\alpha(x) \thicksim 1$, даёт только функция Бесселя с индексом $n = 0$. Здесь пока не учитывался градиентный член пропорциональный $\vec{\theta}$, таким образом из оставшихся фазовых множителей можно выписать условия на индекс $m$. Они определяются нулями в аргументах соответствующих фаз или $m = -1$ и $m = 0$, оба оставшихся члена пропорциональны $K/\gamma$. 

Теперь вернёмся к градиентному члену, вклад от которого занулиться при усреднении по длине ондулятора при $n = 0$, этот вклад даст ненулевой вклад при $n = 1 - 2m$, таким образом в ход пойдут следующие члены разложения $J_m(v)$. Однако, помня интересующий нас диапазон углов, члены разложения будут порядка $\theta_x v^m$, очевидно, что их вклады пренебрежимо малы, и вклад токового члена $\vec{e}_x$ будет доминирующем. Учитывая вышесказанные приближения, перепишем~\ref{eq:field_in_parax_Bessel}:
\begin{equation}
	\label{eq:field_dist_in_integral}
	\begin{array}{lcl}
		\vec{\widetilde{E}}_{\bot}(z_0,  \vec{r}_{\bot 0}, \omega) =
		\cfrac{\omega e}{2c^2z_0}\cfrac{K}{\gamma}\exp[i\cfrac{\omega \theta^2 z_0}{2c}]
		\bigg(J_1(v) - J_0(v)\bigg)\vec{e}_x\times\\
		\\
		\displaystyle\int\limits_{-\lambda_w N/2}^{\lambda_w N/2} dz'
		\exp[i\bigg(k_w \cfrac{\Delta\omega}{\omega_r} + 
		\cfrac{\omega\vec{\theta}^2}{2c}\bigg)z'],
	\end{array}	
\end{equation}
Интеграл легко берётся:
\begin{equation}
	\label{eq:field_dist_nonNorm}
	\begin{array}{lcl}
		\vec{\widetilde{E}}_{\bot}(z_0,  \vec{r}_{\bot 0}, \omega) =
		\cfrac{\omega eL}{c^2z_0}\cfrac{K}{\gamma}A_{JJ}\exp[i\cfrac{\omega \theta^2 z_0}{2c}]
		\sinc \bigg[\bigg(k_w \cfrac{\Delta\omega}{\omega_r} + 
		\cfrac{\omega\vec{\theta}^2}{2c}\bigg)L/2 \bigg ]\vec{e}_x ,
	\end{array}	
\end{equation}
где введено обозначение: $A_{JJ} = J_1(v) - J_0(v)$. В итоге, получено распределение поля в $r\omega$-пространстве. 

Перепишем получившиеся соотношения в нормализованных единицах:
\begin{equation}
	\label{eq:norm_units}
	\begin{array}{lcl}
		\hat{E}_{\bot} = \cfrac{c^2z_0\gamma \widetilde{E}_{\bot}}{e\omega KLA_{JJ}}\\
		\\
		\hat{\theta} = \theta\sqrt{\cfrac{\omega L}{c}}\\
		\\
		\hat{z} = \cfrac{z}{L} ,
	\end{array}	
\end{equation}
а также, 
\begin{equation}
	\hat{C} = CL = 2\pi N\cfrac{\Delta\omega}{\omega_r}
\end{equation}
Таким образом~\ref{eq:field_dist_nonNorm} и~\ref{eq:field_dist_in_integral} могут быть переписаны следующим образом:
\begin{equation}
	\label{eq:field_dist_in_integral}
	\begin{array}{lcl}
		\hat{E}_{\bot} = e^{i\Phi}
		\displaystyle\int\limits_{-1/2}^{1/2} dz'
		\exp[i\bigg(\hat{C} + 
		\cfrac{\hat{\theta}^2}{2}\bigg)z'],
	\end{array}	
\end{equation}

\begin{equation}
	\label{eq:field_dist_Norm}
	\begin{array}{lcl}
		\hat{E}_{\bot} = e^{i\Phi}
		\sinc\bigg(\cfrac{\hat{C}}{2} + 
		\cfrac{\hat{\theta}^2}{4}\bigg).
	\end{array}	
\end{equation}

\begin{figure}
	\centering  
	\begin{minipage}{0.49\textwidth}
		\centering
		\includegraphics[width=\textwidth]{pic/angleC_neg.pdf}
		\caption{Угловое распределение поля при отрицательной сдвижке частоты}
		\label{fig:angle_dist_C_neg}
	\end{minipage}\hfill
	\begin{minipage}{0.49\textwidth}
		\centering
		\includegraphics[width=\textwidth]{pic/angleC_pos.pdf}
		\caption{Угловое распределение поля при положительной сдвижке частоты}
		\label{fig:angle_dist_C_pos}
	\end{minipage}    
\end{figure}
\begin{figure}[h!]
	\centering
	\includegraphics[width=.99\textwidth]{{pic/spec_integ_ang}.pdf}
	\caption{Проинтегрированный по углам спектр излучения. За $\hat{\theta}$ в легенде обозначены пределы интегрирования по углам} 
	\label{fig:spec_integrate_angle}
\end{figure}
На рис.~\ref{fig:angle_dist_C_neg} и рис.~\ref{fig:angle_dist_C_pos} изображены угловые распределения излучения. Их структуру можно понять из рисунка~\ref{fig:traj}. Конструктивная интерференция наблюдается на оси, где есть максимум интерференционной картины на резонансной частоте. Если произвести отрицательную сдвижку по частоте, то выполнение условия конструктивной интерференции: $n \lambda_{ph} = s_{ph} - \lambda_u \cos\theta$ будет наблюдаться при ненулевых углах наблюдения. Обратно, при положительной сдвижке частоты, интенсивность быстро падает, условие резонанса не может выполниться при меньших длинах волн на ненулевых углах, потому что в набег фазы на каждом периоде ондулятора не укладывается целое число длин волн соответствующей гармоники излучения. Говорят, что электрон на каждом периоде ондулятора интерферирует сам с собой. Естественно, говорят о интерференции излучения, которое на оси обгоняет электрон на одну длину волны (или большее число волн, т.е. 1, 2, 3 и т.д.). На следующем периоде ондулятора, электрон снова излучает в фазе с излучённой на прошлом периоде волной. Важной характеристикой в приложениях является проинтегрированный по углам $\hat{\theta}$ спектр излучения, см. рис.~\ref{fig:spec_integrate_angle}. У спектра появляется широкий "хвост". Диапазон углов по которым ведётся интегрирования и единицы измерения для конкретной задачи должны обсуждаться отдельно.  
\begin{figure}[htbp]
	\centering
	\includegraphics[width=.8\textwidth]{{pic/traj}.pdf}
	\caption{Ондулятор как интерференционное устройство} 
	\label{fig:traj}
\end{figure}

\section{Излучение высших гармоник}
\subsection{Амплитудный спектр высших гармоник ондуляторного излучения в зависимости от параметра ондуляторности}
В этом разделе будет дано описание свойств излучения высших гармоник. Начнём с объяснения амплитудного спектра ондуляторного излучения. Понимание данного вопроса необходимо в виду того, что выбор конкретных параметров ондулятора, обычно говорят о параметре ондуляторности $K$, чрезвычайно важен для приложений. Выбор этого параметра напрямую влияет на состав спектра излучения и его амплитудное распределение. Следуя выкладками~\ref{eq:field_dist_nonNorm}, где было введено обозначение $A_{JJ}$, и общей формуле для произвольной гармоники из \cite{wiedemann2015particle} можно написать:
\begin{equation}
	\label{eq:A_JJ}
	A_{JJ}(K) = \cfrac{n^2 K^2}{(1 + K^2/2)^2} \bigg[ J_{\frac{1}{2}(k-1)}\bigg(\cfrac{nK^2}{4 + 2K^2}\bigg) - J_{\frac{1}{2}(k+1)}\bigg(\cfrac{nK^2}{4 + 2K^2}\bigg)\bigg]^2,
\end{equation}
\begin{figure}[h]
	\centering 
	\begin{minipage}{0.99\textwidth}
		\centering
		\includegraphics[width=.79\textwidth]{{pic/A_K}.pdf}
		\caption{Амплитудный спектр гармоник в зависимости от параметра ондуляторности $K$} 
		\label{fig:A_K}
	\end{minipage}
\end{figure}

Графическое представление этой формулы в зависимости от параметра $K$ показано на рис.~\ref{fig:A_K}. Спектр наглядно показывает зависимость амплитуд гармоник от параметра ондуляторности. На ондуляторах, где планируется работать на низких гармониках (1 - 7), преимущественно выбираются малые $K < 2$, если же стоят задачи использовать более высокие гармоники, то параметр $K$ выбирают в районе $2 - 2,5$.
\begin{figure}[h!]
	\begin{minipage}{0.49\textwidth}
		\centering
		\includegraphics[width=\textwidth]{pic/spec_und_1-1.pdf}
		\caption{Спектр ондулятора с ондуляторностью $K = 2,5$}
		\label{fig:spec_und_1-1}
	\end{minipage}
	\begin{minipage}{0.49\textwidth}
		\centering
		\includegraphics[width=\textwidth]{pic/spec_und_1-2.pdf}
		\caption{Спектр ондулятора с ондуляторностью $K = 1$}
		\label{fig:spec_und_1-2}
	\end{minipage}    
\end{figure}

На рис.~\ref{fig:spec_und_1-1} и рис.~\ref{fig:spec_und_1-2} представлены примеры спектров ондуляторного излучения электронного пучка с бесконечно малым эмиттансом. Рисунки наглядно поясняют соображения изложенные выше по амплитудному составу ондуляторного спектра. Уже при при $K = 2,5$ максимум амплитуды приходиться на $7$-ую гармонику.

\section{Заключение к главе}
В главе был дан последовательный вывод свойств ондуляторного излучения, что даёт начальное представление необходимое для проектирования пользовательских станций. Второй подход, который обычно используется в расчёте излучения релятивистского электрона основывается на использовании известных выражений для потенциалов Лиенара — Вихерта в $rt$-пространстве:
\begin{equation}
	\label{eq:Lienard_Wiechert}
	\vec{{E}}_{\bot}(\vec{r}{_0}, t) = -e \cfrac{\vec{n} - \vec{\beta}}{\gamma^2 (1 - \vec{n}\cdot\vec{\beta})^3 |\vec{r}_{ 0} - \vec{r'}|^2} - \cfrac{e}{c}\cfrac{\vec{n}\times[(\vec{n} - \vec{\beta})\times\dot{\vec{\beta}})]}{(1 - \vec{n}\cdot\vec{\beta})^3 |\vec{r}_{ 0} - \vec{r'}|}
\end{equation}
Взяв Фурье-преобразование от этого выражения по времени и совершив преобразования, описанные в \cite{geloni2006fourier}, несложно получается следующее выражение, которое является отправной точкой численных расчётов многих современных симуляционных кодов:
\begin{equation}
	\label{eq:field_dist_in_integral}
	\begin{array}{lcl}
	\vec{\widetilde{E}}_{\bot}(\vec{r}_0, \omega) = 
	-\cfrac{i\omega e}{c}\displaystyle\int\limits_{-\infty}^{\infty} dt'\bigg[\cfrac{\vec{\beta} - \vec{n}}{|\vec{r}_0 - \vec{r'}_0(t')|} - \cfrac{ic}{\omega}\cfrac{\vec{n}}{|\vec{r}_0 - \vec{r'}_0(t')|^2}\bigg]\times\\
	\\
	\exp\bigg[i\omega\bigg(t' + \cfrac{|\vec{r}_0 - \vec{r'}_0(t')|}{c}\bigg)\bigg]
	\end{array}
\end{equation}

Наиболее известный из них --- это SRW, разработанный Олегом Чубарём, \cite{chubar1998proceedings}, \cite{chubar1998accurate}. Код написан на языке $\texttt{C++}$ и является открытым кодом, что добавляет широкие возможность к адаптации кода к пользовательским задачам. Методы кода позволяют рассчитывать излучение релятивистского электрона, с учётом конечности эмиттанса, и далее пропускать получившиеся излучения через оптическую систему с применением подходов Фурье-оптики.

Также нашли широкое применение коды и программы для расчёта синхротронного излучения: SPECTRA \cite{SPECTRA}, XRT (XRayTracer) \cite{XRayTracer}. SPECTRA позволяет рассчитывать спектры излучения из вставных устройств с широкими возможностями в выборе параметров, программа имеет доступный GUI интерфейс, и поэтому легка в использовании. XRT также имеет широкие возможности по моделированию источников синхротронного излучения, рентгенооптических трактов и оптических элементов пользовательских станций.

В работе, в основном, использовались два кода, --- SRW и SPECTRA, код XRT оставлен в стороне, т.к. возможности кода SRW вполне покрывают все потребности в расчётах, код является надёжными и проверенным инструментом при проектировании синхротронных источников. 








