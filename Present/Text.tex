Slide 1
    "Good morning. My name is Alexander and today i will tell you about the study the $e^{+}e^{-} \to K_{S}K_{L}\pi^{0}$ process with the CMD-3 detector."
    
Slide 2
    "On this slide you can see the outline of my presentation. So, I will follow this plan. First I'll give short introduction to the subject of my research, then describe applied methods and discuss obtained results."
    
Slide 3
    "The main goal is to measure the process cross-section. It is very important for the researching the light quarks interaction. Scientists can't predict and estimate the theoretical values in low energy limit because calculation is very hard and QCD is non-perturbative in this region. Current work make the contribution to the new physics discovering connected with anomalous magnetic moment."
    
Slide 4
    In the left side of this slide the VEPP-2000 scheme is shown on. VEPP-2000 is a collider with opposite electron and positron beams. Today this figure is obsolete and now the VEPP-2000 doesn't have the first two complexes. In 2017 this complexes were combined into one.
    In the right side the CMD-3 detector is presented. CMD-3 is cryogenic magentic detector which consists of the drift chamber, Z chamber, BGO calorimeter, LXe calorimeter, CsI calorimeter and superconductive magnets.
    In this work we use data collected in 2011-2012 seasons and integrated luminosity is 33 reverse pb."
    
Slide 5
    "So, for study we use the $K_{S} \to \pi^{+} \pi^{-}$ and $\pi^{0} \to \gamma \gamma$ decay modes. Charged pions are detected in drift chamber. Hence we take events with two tracks. Also this tracks must have one common vertex. And number of photons sould be more than one. In the right side the predicted Feyman diagram is presented."
    
Slide 6
    "On this slide you can see the list of the criteria for the event selection. I want to emphasise that the photon born in the vacuum chamber has 98\% to interact with LXe calorimeter." 
    
Slide 7
    "Ionization looses of the tracks have to consistency to charged pions. For example, this os Kaon... So, we fit this form like this function. The a and b are parameters of this function..."
    
Slide 8
    "After than we calculate a new criterion $\xi$ and build the next plot: dependence the negative charged xi on the positive charged xi. You can see that the simulation doesn't calibrate and this problem has a consequence to the efficiency."
    
Slide 9
    "Distribution of the cosine between the momentum and the radius-vector of the $K_{S}$ meson in the XY plane. The selection criteria is this inequality. Look at scheme in this part. This is the radius-vector which I continue and this is the momentum."
    
Slide 10-12
    "I'm sorry, but I don't have enough time to discuss it and I will hust show you the next three slides."
    
Slide 13
    "On this slide the dependence of the $K_{S}$ meson mass on the $\pi^{0}$ meson mass is shown. On the left side the simulation results are and on the right side the experimental results are. Look at this part of the experimental results. The $e^{+} e^{-} \to \pi^{+} \pi^{-} \pi^{0} \pi^{0}$ process condense the points. So, we must reduce the background influence."
    
Slide 14
    "The efficiency is calculated as ratio of the signal events to the total number."
    
Slide 15
    "For reducing the background influence we fit the $K_{S}$ mass distibution as a sum of three Gauss and a linear trend. The area of the linear line is the background."
    
Slide 16
    "Finally, we determine the cross-section as this equation. So, on the figure I present the comparison of three collaborations."
    
    