{\actuality}
Данная работа посвящена разработке рентгенооптических трактов синхротронного источника ЦКП <<СКИФ>> --- Центр Коллективного Пользования <<Сибирский кольцевой источник фотонов>>. За последние три десятилетия мир увидел активное развитие специализированных источников синхротронного излучения и соответствующих методов исследования вещества с использованием синхротронного излучения в рентгеновском диапазоне. Главные параметры излучения, который достигаются на данных установках являются: высокий поток фотонов, направленность излучения в малый телесный угол, когерентность. Эти параметры крайне необходимы для проведения качественных экспериментов с революционными результатами в области химии, биологии, материаловедении, медицины и многих других отраслях науки и техники, \cite{willmott2019introduction}.

Высокая востребованность данной работы заключается в том, что отечественная наука претерпевает стагнацию в области развития специализированных источников рентгеновского излучения. Проектируемый в Новосибирске синхротронный источник является первым на территории России специализированным источником с проектными параметрами не уступающими мировым установкам, таким, например, как: MAX-IV \cite{max2010max}, NSLS-II \cite{NSLS}, PETRA-III \cite{balewski2004petra}, Diamond \cite{DIAMOND} и д.р., а по некоторым данным с запасом превосходящих их\cite{zorin2019radiation}.

\textbf{Цель} данной работы --- разработка проекта станций первой очереди, вставными устройствами на которых являются сверхпроводящие ондуляторы. Это станции: 1-1 --- <<Микрофокус>>, 1-2 --- <<Структурная диагностика>>, 1-4 --- <<XAFS-спектроскопия и магнитный дихроизм>>.  

В разработку проекта входит ряд \textbf{задач}, которые выполнены в данной работе:
\begin{itemize}

	\item Расчёт ондуляторного излучения с помощью численного моделирования, получение спектров и сечений пучка из указанных устройств, максимально обективно описывающих реальное излучение.
	\item Разработка оптических трактов: расчёт тепловых нагрузок, расчёт спектров и сечений пучка после прохождение оптических элементов. 
	\item Разработка программного кода для реализации выше приведённых задач и удобному воспроизведению результатов расчётов любым участником проекта.
\end{itemize}

В работе даётся изложение теории ондуляторного излучения, необходимое для понимания основных моментов при проектировании станций, и далее по тексту приводятся результаты расчётов и основные концептуальные идеи по реализации пользовательских станций ЦКП <<СКИФ>>.


