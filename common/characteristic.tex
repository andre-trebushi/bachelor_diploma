{\actuality}
С начала 90-ых годов мир увидел активное развитие специализированных источников синхротронного излучения и соответствующих методов исследования вещества с его использованием, \cite{willmott2019introduction}. Основные параметры, которые важны для пользователей: величина потока фотонов, спектральное распределение, когерентность, временная структура, концентрация излучения в малом телесном угле. Перечисленные свойства излучения определяются ускорительным комплексом и вставными устройствами. Именно от параметров электронного пучка и конструкции вставного устройства зависят характеристики излучения, падающего на образец, они же учитываются при планировании исследований.

Данная работа посвящена расчёту оптических трактов экспериментальных стаций синхротронного источника --- \underline{Ц}ентр \underline{к}оллективного \underline{п}ользования <<\underline{С}ибирский \underline{к}ольцевой \underline{и}сточник \underline{ф}отонов>> (ЦКП <<СКИФ>>). 
%В связи с поручением президента РФ о создании синхротронного источника излучения в г. Новосибирск, ведутся работы по проектированию ЦКП <<СКИФ>>. 
ЦКП <<СКИФ>> будет специализированным источником синхротронного излучения с проектными параметрами лучше, чем у лидирующих мировых установок\cite{zorin2019radiation}, таких, например, как: MAX-IV \cite{max2010max}, NSLS-II \cite{NSLS}, PETRA-III \cite{balewski2004petra}, Diamond \cite{DIAMOND} и д.р.. Существенное внимание в работе уделяется оптическим свойствам рентгеновского излучения в трактах экспериментальных станций, поскольку именно ими определяются предельные возможности экспериментальных методов исследования вещества. 

Настоящая работа заключается в исследовании оптических трактов трёх экспериментальных станций объекта <<СКИФ>>: 1-1 --- <<Микрофокус>>, 1-2 --- <<Структурная диагностика>>, 1-4 --- <<XAFS-спектроскопия и магнитный дихроизм>>. В качестве источников излучения на них используются сверхпроводящие ондуляторы. 

\textbf{Цель} работы --- разработка концептуального проекта оптической части экспериментальных станций:\\
Для достижения поставленных целей необходимо было решить ряд \textbf{задач}:
\begin{itemize}
	\item расчёт ондуляторного излучения с помощью численного моделирования, получение спектров и сечений пучка из указанных устройств;
	\item предложение о создании специального ондулятора с уширенными гармониками, и расчёт параметров пучков излучения из него;
	\item проектирование оптических трактов: расчёт тепловых нагрузок на оптические элементы, расчёт сечений пучка после прохождения оптических элементов;
	\item разработка программного кода для реализации вышеприведённых задач и удобного воспроизведения результатов расчётов.
\end{itemize}

%Математические методы, при помощи которых осуществляется моделирование пользовательских станций: от получения сечения и спектра излучения из вставного устройства до распространения этого излучения через оптическую систему. 
Следует отметить, что в конце 90-ых годов широкое применение нашли программные коды, которые решают задачу об излучении релятивистского электрона в магнитном поле из первых принципов --- уравнений Максвелла. Наиболее популярным кодом является код SRW (\underline{S}ynchrotron \underline{R}adiation \underline{W}orkshop), разрабатываемый Олегом Чубарём, \cite{SRW} - \cite{chubar1998proceedings}. В своих работах \cite{chubar1998accurate} - \cite{chubar2002physical}, автор кода предоставляет исчерпывающие пояснения к методу расчёта излучения релятивистского электронного пучка, используемого в коде SRW. В серии работ \cite{chubar2001wavefront} -%  \cite{del2014proposal}, \cite{canestrari2014partially}, \cite{sutter2014perfect}, \cite{suvorov2014partially} 
\cite{idir2017alignment} рассматривается моделирование оптических элементов от дифракции на идеальных кристаллах до оптимизации системы фокусирующих зеркал. В целом, код SRW предоставляет обширный набор инструментов для моделирования источников синхротронного излучения. 

В первой главе работы даётся изложение теории ондуляторного излучения,
%, необходимое для понимания основных моментов при проектировании экспериментальных станций, 
во второй главе приводятся результаты расчётов в рамках задач этой работы и основные концептуальные идеи по реализации пользовательских станций ЦКП <<СКИФ>>.


